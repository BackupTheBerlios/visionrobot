\chapter{Procesamiento de texto reconocido}

\section{Introducci�n}
Una vez que el OCR ha reconocido una cadena de texto y la ha validado, env�a la se�al a este m�dulo, que se encarga de procesarlo, y crear una nueva salida en el formato propio del pipeline para el entorno 3D y el robot.

\section{Detalles}
\begin{itemize}
  \item {\bf Entrada}: Una cadena ya preparada del OCR.
  \item {\bf Salida}: Un resultado en forma de cadena de respuesta o de orden para el robot.
  \item {\bf Descripci�n}: M�dulo que, a partir de una cadena ya reconocida, calcula un resultado matem�tico, o da �rdenes de conducta a un robot.
\end{itemize}

\section{Implementaci�n}

Para implementarlo, hemos usado un programa en \textbf{Prolog}, que est� formado en su mayor parte por una DCG \footnote{Definite clause grammar}. Este programa recibe la entrada del m�dulo, y, tras procesarla, crea una salida, que puede ser una cadena de respuesta, el resultado de una operaci�n aritm�tica (soporta par�ntesis y anidamiento de expresiones) o una orden para la salida.

\section{Ampliabilidad}

La idea de crear el m�dulo como antes hemos explicado ha sido propiciada por la idea de que la ``inteligencia del robot'' fuese algo ampliable. Con s�lo modificar este archivo, se puede dotar a la m�quina de m�s informaci�n (ampliando el diccionario), o de m�s capacidad de proceso (creando m�s reglas de reconocimiento en la DCG).
