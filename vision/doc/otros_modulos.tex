\chapter {Otros m�dulos}

\section{Introducci�n}
A continuaci�n explicamos el funcionamiento del resto de los m�dulos. Los aglutinamos en esta secci�n, por carecer de inter�s la explicaci�n detallada de los mismos, debido a su sencillez.

% TODO: poner las referencias

\section {M�dulo de post-gesti�n}
Hemos a�adido un m�dulo que recoja todas las se�ales que son generadas por las distintas ramas de proceso del \emph{pipeline}, y las agrupe en una sola salida v�lida, para ofrecer de este modo, a trav�s de sus puertos, datos a todos los m�dulos de salida.

Puede examinar su documentaci�n de c�digo en \ref{caca}.

\section{Ventana de par�metros}
La ventana de par�metros es una interfaz gr�fica en la que es posible elegir un color, y tres tolerancias (una para el rojo, otra para el verde y otra para el azul), y alimentar as� a un m�dulo de filtro, de tal modo que la parametrizaci�n de los valores del mismo se pueda hacer en tiempo real.

Puede examinar su documentaci�n de c�digo en \ref{caca}.

\section{Ventana de im�genes}
La ventana de im�genes es simplemente un m�dulo que muestra gr�ficamente, en una ventana que se redimensiona autom�ticamente, una imagen que sigue el formato propio de nuestra aplicaci�n (ver secci�n \ref{formato_imagenes}).

La ventana de im�genes tiene como funci�n a�adida la capacidad de guardar una toma de la imagen que se este dibujando en el instante en el que se pulsa \textbf{F5}.

Puede examinar su documentaci�n de c�digo en \ref{caca}.

\section{M�dulo de salida}
Este m�dulo es simplemente una ventana con un cuadro de texto que es capaz de imprimir en �l todas las cadenas (en formato C, \verb|char *|) que le llegan a trav�s de sus puertos.

Puede examinar su documentaci�n de c�digo en \ref{caca}.