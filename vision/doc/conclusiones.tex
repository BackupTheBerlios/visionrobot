\section{Conclusiones}

Desde que comenzamos y hasta la finalizaci�n completa del proyecto, hemos pasado por varias etapas. En una primera fase, afrontamos la aplicaci�n como algo ligeramente borroso, no demasiado definido. Poco a poco fue tomando forma, proceso en el cual fuimos haci�ndonos conscientes de algunos puntos clave en el desarrollo de un proyecto como �ste. Los m�s importantes que podemos esquematizar como conclusiones, son los que detallamos a continuaci�n:


\subsubsection{El proceso de desarrollo no ha de ser tomado como un bloque global que atacar directamente}
Esto s�lo lleva a un c�digo confuso y muy acoplado, que da muy poca oportunidad a la expansi�n o a la reutilizaci�n del mismo, incluso dentro del proyecto. Hemos tenido la suerte de hacernos cargo de esta circunstancia en una fase poco avanzada del proyecto, e invertir un esfuerzo extra en organizar las cosas, dividir el trabajo a conciencia, con sentido, y asignando a cada cual las partes en las que m�s puede desarrollar sus habilidades. Asimismo, todo el tiempo invertido en realizar una buena arquitectura de trabajo, nos ha dado alas para ampliar la aplicaci�n en la medida en que las necesidades del proceso fueron ampli�ndose.

\subsubsection{El cumpliminento de los requisitos del proyecto ha de ser el objetivo primario}
Con esto queremos se�alar que, a pesar de que todo el proyecto se daba, como idea y como realidad, de una manera muy importante a la experimentaci�n y a las pruebas, nuestra principal prioridad era cumplir con los requisitos primeros que estaban impuestos como fin del proceso de desarrollo. De esta manera, conseguimos poner punto y final al proceso de implementaci�n bastante antes de la fecha l�mite, pudi�ndonos dedicar al desarrollo de una memoria concreta, explicativa, y que abarcara todos los aspectos que hemos considerado importantes, as� como a perfeccionar esos detalles que siempre quedan como \textbf{objetivos secundarios}, pero que tanto realzan el contenido final de la aplicaci�n.

\subsubsection{Es posible llevar a cabo las ideas que han conducido a la motivaci�n de la realizaci�n de un proyecto}
Decidimos trabajar en la visi�n por computador y en rob�tica porque ten�amos ideas que hab�amos madurado durante alg�n tiempo, y quer�amos llevarlas a cabo. Tras la finalizaci�n del trabajo, nos hemos percatado de que teniendo un m�todo serio de trabajo y aprendizaje, siendo consecuentes con los plazos de entrega que nosotros mismos nos impusimos, y trabajando con tes�n, es posible conseguir plasmar en un proyecto de software (y ligeramente, hardware), como es el nuestro, aquellas ideas que hab�amos tenido. Eso s�, perfeccion�ndolas y eliminando de las mismas todas aquellas partes que eran imposibles en el marco de un a�o de trabajo.


\subsubsection{La visi�n por computador es un campo en el que queda mucho por hacer}

Si bien es verdad que no hemos llegado ni mucho menos a lo m�s profundo, innovador e interesante de este campo, s� es cierto que podemos decir, tras haber estado un a�o experimentando y trabajando, que la visi�n por computador tiene mucho por delante a�n. A nuestro nivel nos hemos dado cuenta, por ejemplo, de lo muy sensible que es el filtrado de im�genes al tipo de luz que incida sobre los objeto. Del reconocimiento de gestos de una imagen fija previamente pensada al reconocimiento en tiempo real hay un abismo en la efectividad de resultados. La visi�n por computador, seg�n estos ejemplos y otros muchos que hemos encontrado es, a nuestro parecer, una ciencia que a�n tiene que desarrollarse mucho.


