\section {Introducci�n}

Esta secci�n ofrece una primera toma de contacto con el proyecto, explicando de la manera m�s simple y concisa posible los primeros detalles generales que hemos considerado de inter�s para el conocimiento de ``Reconocimiento visual de instrucciones''.

\subsection{Objetivos}

El objetivo del proyecto consiste desarrollar en un motor modular de visi�n no dependiente del entorno y el reconocimiento de mensajes, a trav�s de las im�genes capturadas desde una c�mara, para la implantaci�n del sistema de �rdenes de un robot. Para la demostraci�n de la funcionalidad se usar�n un entorno en 3D y un robot construido desde cero.

La idea fundamental ha sido partir desde cero y emplear una gran parte de los conocimientos adquiridos durante la carrera para desarrollar un motor de reconocimiento de visi�n, con unos resultados �tiles, diferenciables, y, en la medida de lo posible, vistosos. Mediante la implementaci�n, adem�s, hemos procurado desarrollar un sistema eficiente y s�lido, a la vez que dotado de gran generalidad, para que el flujo de producci�n de la aplicaci�n y las posteriores modificaciones fuesen lo m�s c�modas y sencillas posibles.

\subsection{Organizaci�n de la memoria}
La memoria est� organizada en tres partes. La primera, en la que nos encontramos (parte \ref{definicion}), se dedica a la explicaci�n del proyecto como tal, con el prop�sito de proporcionar una visi�n general y directa de {\em Reconocimiento visual de instrucciones}. Aqu� explicamos la arquitectura de ``tuber�a'' de la aplicaci�n, y el posible uso futuro del trabajo realizado, entre otras cosas. Esta primera parte es la base para adquirir una visi�n general y completa del trabajo, necesaria tanto para conocerlo como para usarlo o ampliarlo.

Como segunda parte de la memoria hemos a�adido un primer anexo (\ref {anexo1}) en el que detallamos los m�dulos que usamos, yendo un nivel m�s abajo en el detalle de explicaci�n de los mismos. Aqu� damos ya una visi�n menos general, pero m�s profunda, de las diferentes tecnolog�as caracter�sticas que hemos usado para desarrollar las diferentes partes de la aplicaci�n. Hemos intentado que este m�dulo de la memoria sirva como base para el uso met�dico del proyecto, y que ofrezca al lector una primera aproximaci�n a la verdadera arquitectura e implementaci�n del trabajo. Por tanto, es de inter�s para aquellos que vayan a usar el proyecto de una manera m�s avanzada que la de la simple prueba.



