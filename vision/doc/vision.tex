\documentclass[a4paper,12pt]{book}
\usepackage[latin1]{inputenc}
\usepackage[spanish]{babel}
\usepackage{graphicx}

\title {Reconocimiento Visual de Instrucciones de un entorno 3D para un Robot Aut�nomo}
\author {Carlos Le�n \and Jorge Mendoza \and Diego S�nchez}

\makeindex

\begin{document}

\maketitle

\thanks {Gracias a Jos� Antonio L�pez por su ayuda con el robot, y a Juan Rodr�guez por su ayuda con Prolog.}

\tableofcontents

% TODO: A ver qu� t�tulo pongo aqu�
\part {Definici�n del proyecto}
\label {definicion}

\section{Resumen del proyecto}

\subsection{Resumen en castellano}
El proyecto consiste en el desarrollo de un motor de visi�n por computador modular, que tiene como objetivo el reconocimiento de gestos de las manos y de �rdenes escritas, adquiridas a trav�s de interfaz de im�genes como una c�mara y otros, y la traducci�n de las mismas a �rdenes inteligibles. Asimismo, se crear�n un microrrobot y un entorno de visualizaci�n en tres dimensiones, que muestren, mediante el movimiento del mismo y el desplazamiento del personaje en el entorno, respectivamente, la ejecuci�n de dichas �rdenes.

\subsection{Briefing in English}
The project consists on the development of a modular machine vision engine, that has as objective the reconoissance of hand gestures and written orders, adquired through an image interface like a camera and others, and the translation of these to understandable orders. Also, a micro-robot and a three dimension environment will be created that show, by the movement and the displacement of the character in the environment, respectively, the execution of the mentioned orders.

\subsection{Palabras clave}
Visi�n, reconocimiento, OCR, neuronal, entorno, 3d, robot, pipeline, m�dulo, imagen.

\chapter {Introducci�n}

\section{Objetivos}

El objetivo del proyecto consiste desarrollar en un motor modular de visi�n no dependiente del entorno y el reconocimiento de mensajes, a trav�s de las im�genes capturadas desde una c�mara, para la implantaci�n del sistema de �rdenes de un robot. Para la demostraci�n de la funcionalidad se usar�n un entorno en 3D y un robot construido desde cero.

\section{Organizaci�n de la memoria}


\section{Resumen del proyecto}


\section{Especificaci�n}
 
Vamos  a exponer a continuaci�n los requisitos y caracter�sticas que definen el proyecto de Visi�n por Computador que hemos desarrollado. Esta enumeraci�n ha sido la base a partir de la cu�l se ha desarrollado el dise�o y la posterior implementaci�n de cada uno de los m�dulos que constituyen el proyecto. Por tanto describe las funcionalidades que establecimos inicialmente como objetivos que se deb�an satisfacer una vez concluido el proyecto. Posteriormente ser� necesario realizar una serie de pruebas que confirmen que cada uno de las siguientes caracter�sticas han sido implementadas adecuadamente para corroborar el �xito del proyecto. Est�s pruebas se detallar�n de forma exhaustiva en el apartado correspondiente.

\subsection{Interacci�n con una c�mara de video }
Se deber� poder inicializar, configurar y utilizar una c�mara que se comunique con nuestro proyecto suministrando de forma adecuada las im�genes que deber�n ser analizadas.

\subsection{Adquisici�n y tratamiento de las im�genes}
Los datos suministrados por la c�mara deber�n ser modificados convenientemente para ajustarse a un formato �til para su manipulaci�n por parte de nuestro proyecto. Adem�s, ser� necesario que la imagen recibida a trav�s de la c�mara sea tratada mediante una serie de filtros que resaltar�n y corregir�n adecuadamente los elementos necesarios para su posterior an�lisis.

\subsection{Tolerancia a diferentes condiciones de iluminaci�n }
Debido a que se trata con elementos visuales (im�genes de una c�mara de video), es necesario que el sistema sea tolerante a las diferentes condiciones de iluminaci�n que se pueden dar en distintos entornos. Se debe proporcionar una soluci�n a este respecto que permita la pertinente graduaci�n de los procesos de filtrado y correcci�n de im�genes para poder adaptarse a unas condiciones ambientales variables (obviamente siempre dentro de unos m�rgenes m�nimos de iluminaci�n).

\subsection{An�lisis de las im�genes }
Una vez tratadas adecuadamente, las im�genes deber�n ser analizadas para poder determinar si corresponden a un item reconocido por el sistema. El sistema deber� reconocer dos tipos de items: gestos realizados por el usuario que se interpretar�n como comandos de navegaci�n del robot, y textos escritos en carteles.
Los comandos se generar�n a partir de los gestos que el usuario realice con sus dos manos. En cada una de ellas deber� portar unos guantes con unos puntos de color diferenciados para cada mano. Los gestos de una de las manos se interpretar�n como �rdenes de navegaci�n del tipo avanzar, girar o parar mientras que los gestos de la segunda mano se interpretar�n como el grado de intensidad con la que la orden actual debe ser ejecutada. As� se podr� ordenar que el robot gire a la izquierda con mayor o menor velocidad de rotaci�n por ejemplo. En cuanto al texto, deber� estar escrito sobre carteles de un color que destaque en el entorno. El texto podr� consistir en operaciones aritm�ticas sencillas, �rdenes de control para la navegaci�n del robot, o sencillas preguntas sobre datos que el sistema posea en su base de datos. Una vez analizado, el sistema dar� respuesta seg�n la naturaleza del texto de entrada. En caso de tratarse de una operaci�n aritm�tica, se deber� dar la soluci�n del c�lculo. Si se trata de una orden de navegaci�n, se deber� considerar como un comando de movimiento para el robot que deber� comportarse en consecuencia. Y por �ltimo, si se trata de una pregunta, se deber� dar una respuesta coherente a la sem�ntica y formulaci�n sint�ctica de �sta.

\subsection{Simulaci�n del comportamiento del robot seg�n los comandos procesados }
Para poder comprobar el correcto funcionamiento del sistema de reconocimiento de gestos, se deber� utilizar un sistema que visualice de forma sencilla e intuitiva una simulaci�n del comportamiento que deber�a presentar el robot en respuesta a las �rdenes recibidas. Esta funcionalidad ser� �til tanto para realizar pruebas como para contar con una salida auxiliar que corrobore el comportamiento de un robot f�sico que se conecte al sistema. De esta manera se podr� ir verificando que cada orden procesada es interpretada de forma similar por el robot f�sico en concordancia con lo visualizado en la simulaci�n.

\subsection{Creaci�n y conexi�n al sistema de reconocimiento de un robot }
Se deber� construir un robot mec�nico con capacidades motrices b�sicas (avanzar, retroceder, y girar en ambos sentidos) que se conectar� al sistema de reconocimiento de gestos y actuar� en consecuencia. 



\chapter{Estado del arte}


\section{Punto de partida}
\emph{Visi�n por computador} es un proyecto nuevo. Ha sido realizado desde cero. 

\section{Conocimientos previos}
Para llevar a cabo el proyecto hemos hecho uso de una gran parte de los conocimientos adquiridos durante los a�os de carrera.


\section{Aportaciones propias}


\chapter{Dise�o y arquitectura del proyecto}

\section{Introducci�n}
En este cap�tulo vamos a dar una idea global y poco exhaustiva, pero concreta, de la arquitectura de \emph{Visi�n por Computador}. Tiene dos apartados importantes: la arquitectura modular, y la conexi�n de estos m�dulos.

\section {Arquitectura de la aplicaci�n. M�dulos en tuber�a}
Al dise�ar la aplicaci�n escogimos implementar un sistema altamente modular para conseguir un alto grado de independencia en el desarrollo y de facilidad de ampliaci�n. Por esta raz�n hemos invertido parte del trabajo en desarrollar una plataforma propia de enlace de m�dulos, en la que la una parte de la aplicaci�n (lo que hemos denominado {\em pipeline} o {\em tuber�a}), se encargue de realizar el trabajo mec�nico, que, b�sicamente, se compone de:
\begin{itemize}
\item Conectar los m�dulos mediante puertos independientes con diferente informaci�n por puerto, pudiendo crear conexiones {\em 1 a 1}, {\em n a 1}, {\em 1 a n}, y {\em n a n}.
\item Iniciar y cerrar los m�dulos, creando y liberando la memoria necesaria y llamando a las funciones pertinentes de cada m�dulo.
\item Gestionar un reloj de ciclos de ejecuci�n, transmitiendo la acci�n por el grafo que forma la arquitectura de m�dulos.
\item Control de proyectos de aplicaci�n din�micos, mediante definici�n de los mismos en {\bf XML}. De esta forma, diferentes archivos de configuraci�n de proyecto puden crear aplicaciones totalmente distintas sin tener que reprogramar nada.
\item Manejo de errores mediante retrollamadas a funciones definidas por el usuario.
\end{itemize}
El {\em pipeline} es multiplataforma y funciona con m�dulos compilados desde {\em cualquier lenguaje est�ndar} como bibliotecas din�micas. Esto dota a la aplicaci�n de un marco muy amplio de uso en cualquier �mbito de desarrollo.


\section{Diagrama de pipeline de ``Visi�n por Computador''}
En la figura \ref{diagrama_vision_computador} hemos esquematizado todo el proceso que siguen los datos de nuestra aplicaci�n hasta llegar a una salida visible por el usuario. Los datos comienzan en la interfaz de im�genes (puede ser una imagen de c�mara, un v�deo, una imagen fija...), y descienden por el grafo de m�dulos hasta el \textbf{robot} o el \textbf{entorno 3D}. Asimismo, tenemos tambi�n de entrada las ventanas de par�metros, que dotan a los filtros de los valores necesarios \emph{en tiempo real}.

Tras el filtrado pertinente de cada imagen, se las lleva a un m�dulo de proceso (redes neuronales para los guantes, y algoritmo de \emph{OCR} para el texto), y, paralelamente, a ventanas de visualizaci�n, para depuraci�n y comprobaci�n de resultados. Se filtran las se�ales err�neas y, para el m�dulo de texto, se env�a la informaci�n a un m�dulo de DCG\footnote{Definite clause grammar} que genera una salida como respuesta ``inteligente''. Cuando la informaci�n ya ha sido extra�da de cada imagen, s�lo queda unificarla con el resto de datos, para dar una salida coherente.
%\begin{figure}
%  \label{diagrama_vision_computador}
%  \centering

  %\includegraphics[width=120mm]{pipeline.eps}

  %\caption{Diagrama de tuber�a}
%\end{figure}

\begin{figure}
  \centering
  \includegraphics[width=13.608cm,height=11.404cm,bb=0 0 1113 854]{pipeline.eps}
  \caption{Diagrama de tuber�a}
  \label{diagrama_vision_computador}
\end{figure}




\section{Conclusiones}

Desde que comenzamos y hasta la finalizaci�n completa del proyecto, hemos pasado por varias etapas. En una primera fase, afrontamos la aplicaci�n como algo ligeramente borroso, no demasiado definido. Poco a poco fue tomando forma, proceso en el cual fuimos haci�ndonos conscientes de algunos puntos clave en el desarrollo de un proyecto como �ste. Los m�s importantes que podemos esquematizar como conclusiones, son los que detallamos a continuaci�n:


\subsubsection{El proceso de desarrollo no ha de ser tomado como un bloque global que atacar directamente}
Esto s�lo lleva a un c�digo confuso y muy acoplado, que da muy poca oportunidad a la expansi�n o a la reutilizaci�n del mismo, incluso dentro del proyecto. Hemos tenido la suerte de hacernos cargo de esta circunstancia en una fase poco avanzada del proyecto, e invertir un esfuerzo extra en organizar las cosas, dividir el trabajo a conciencia, con sentido, y asignando a cada cual las partes en las que m�s puede desarrollar sus habilidades. Asimismo, todo el tiempo invertido en realizar una buena arquitectura de trabajo, nos ha dado alas para ampliar la aplicaci�n en la medida en que las necesidades del proceso fueron ampli�ndose.

\subsubsection{El cumpliminento de los requisitos del proyecto ha de ser el objetivo primario}
Con esto queremos se�alar que, a pesar de que todo el proyecto se daba, como idea y como realidad, de una manera muy importante a la experimentaci�n y a las pruebas, nuestra principal prioridad era cumplir con los requisitos primeros que estaban impuestos como fin del proceso de desarrollo. De esta manera, conseguimos poner punto y final al proceso de implementaci�n bastante antes de la fecha l�mite, pudi�ndonos dedicar al desarrollo de una memoria concreta, explicativa, y que abarcara todos los aspectos que hemos considerado importantes, as� como a perfeccionar esos detalles que siempre quedan como \textbf{objetivos secundarios}, pero que tanto realzan el contenido final de la aplicaci�n.

\subsubsection{Es posible llevar a cabo las ideas que han conducido a la motivaci�n de la realizaci�n de un proyecto}
Decidimos trabajar en la visi�n por computador y en rob�tica porque ten�amos ideas que hab�amos madurado durante alg�n tiempo, y quer�amos llevarlas a cabo. Tras la finalizaci�n del trabajo, nos hemos percatado de que teniendo un m�todo serio de trabajo y aprendizaje, siendo consecuentes con los plazos de entrega que nosotros mismos nos impusimos, y trabajando con tes�n, es posible conseguir plasmar en un proyecto de software (y ligeramente, hardware), como es el nuestro, aquellas ideas que hab�amos tenido. Eso s�, perfeccion�ndolas y eliminando de las mismas todas aquellas partes que eran imposibles en el marco de un a�o de trabajo.


\subsubsection{La visi�n por computador es un campo en el que queda mucho por hacer}

Si bien es verdad que no hemos llegado ni mucho menos a lo m�s profundo, innovador e interesante de este campo, s� es cierto que podemos decir, tras haber estado un a�o experimentando y trabajando, que la visi�n por computador tiene mucho por delante a�n. A nuestro nivel nos hemos dado cuenta, por ejemplo, de lo muy sensible que es el filtrado de im�genes al tipo de luz que incida sobre los objeto. Del reconocimiento de gestos de una imagen fija previamente pensada al reconocimiento en tiempo real hay un abismo en la efectividad de resultados. La visi�n por computador, seg�n estos ejemplos y otros muchos que hemos encontrado es, a nuestro parecer, una ciencia que a�n tiene que desarrollarse mucho.



\chapter{Trabajo futuro}


\section{Enlace con otros grupos}


\section{Continuidad del proyecto}


\section{Pruebas}

\subsection{Pruebas de arquitectura}

\subsubsection{Pruebas de desarrollo}
Uno de los mayores retos del proyecto ha sido conseguir una arquitectura s�lida que satisficiera nuestras necesidades de conexi�n. La evoluci�n de la generaci�n de la aplicaci�n ha llevado impl�citas las pruebas diarias del �xito del dise�o y la implementaci�n de nuestro \emph{pipeline}. Por eso detallamos los cambios del mismo, como resultado de nuestros experimentos.

La primera visi�n del dise�o global era mucho m�s simple que la que finalmente hemos acabado usando: los m�dulos s�lo se conectaban en forma de �rbol. Pronto tuvimos que abandonar este enfoque, pues los requisitos de conexi�n de m�dulos se mostraron m�s complejos de lo que estimamos en un principio.

As� pues, pensamos en conectar los m�dulos \emph{1 a 1} en forma lineal. Los primeros resultados fueron satisfactorios: los m�dulos se comunicaban una vez que la implementaci�n del dise�o dej� de tener errores. La implementaci�n en este punto comenzaba a ser s�lida, y pronto ampliamos el dise�o para que los m�dulos tuvieran conexiones de \emph{1 a N}. Con esto conseguimos, por ejemplo, ver las im�genes que generaban los filtros sin tener que cambiar en absoluto los m�dulos que operaban. El dise�o comenzaba a dar sus frutos, empez�bamos a ahorrar horas de trabajos y a reutilizar fuertemente el c�digo desarrollado, ya que un m�dulo de ``ventana de im�genes'' pod�a, instanci�ndose varias veces, presentar diferentes im�genes.

El dise�o ya comenzaba a ser realmente s�lido, sin embargo, nos vimos en la necesidad de que un m�dulo aceptase varias entradas. Por tanto, a�adimos esa funcionalidad. En esta fase del dise�o completamos casi toda la aplicaci�n. Las pruebas y la correcci�n de errores fueron paralelas y terminaron por dar con un conjunto muy fiable, con conexiones \emph{N a N}.

En �ltima instancia, nos dimos cuenta de que el sistema de puertos no era del todo completo. Un m�dulo s�lo se pod�a conectar a otro por un puerto. Esto nos presentaba el inconveniente en el m�dulo de c�lculo de respuesta, que deb�a ofrecer la salida de la orden y el par�metro de forma independiente. Finalmente, pues, a�adimos m�s potencia a los puertos, dotando a la estructura de una completitud amplia y s�lida.

Como ejemplo global de pruebas y de la utilidad de la arquitectura de \emph{pipeline}, podemos rese�ar el del m�dulo de ``gesti�n''. Poco valorado al principio, pensamos que iba a ser un simple tr�mite de la salida global. Sin embargo, finalmente el diagrama total tiene 4 instancias del m�dulo, para las cuales no hemos tenido que reprogramar nada, s�lo variar el archivo de proyecto XML.

\subsubsection{Eficiencia}

El \emph{pipeline} no es un ejemplo de velocidad de proceso. Las pruebas que hemos realizado nos han permitido, en un ordenador con un microprocesador \emph{Intel Pentium IV} con una frecuencia de reloj de $3.0$ GHz, alcanzar velocidades de ciclo de 200 milisegundos. Para una aplicaci�n del �mbito docente como es la que presentamos, el resultado es desde luego m�s que suficiente, pero no deja de ser un tiempo de ejecuci�n lento para requisitos como, por ejemplo, de tiempo real.

\subsubsection{Comprobaci�n de los m�dulos}
A continuaci�n detallamos las pruebas del aspecto arquitect�nico de los m�dulos:
\begin{itemize}
\item \textbf{Generaci�n im�genes y ventana de im�genes}: El resultado esperado de este m�dulo era la generaci�n de im�genes desde diversas fuentes, en un formato unificado. Para esto, usamos principalmente la ventana de im�genes, comprobando que las im�genes correspondiesen a lo esperado. Para ejemplos de resultados, puede remitirse a la secci�n \ref{imagenes_ejemplos_graficos}.

\item \textbf{Filtros}: La serie de transformaciones que sufre la imagen en el m�dulo de filtro (gestos o carteles) es el resultado de un largo periodo de investigaci�n sobre im�genes de prueba.

La elecci�n de una transformaci�n u otra ha estado dirigida siempre con el prop�sito de conseguir un an�lisis posterior mucho mas simple y fiable.

Todos los filtros que fuimos creando que no presentaban valor a�adido han sido eliminados, quedando as� los m�nimos necesarios para facilitar la extracci�n de informaci�n de la imagen en el posterior an�lisis (red neuronal, OCR).

La elecci�n del tama�o de las mascaras utilizadas o la creaci�n de varios filtros dentro de los mismos bucles se ha realizado siempre con la intenci�n de que la fase de filtro sea lo mas r�pida posible, no impidiendo que la aplicaci�n funcione en tiempo real.

\bigskip
Tiempo aproximado del filtro de gestos:  5 milisegundos.

Tiempo aproximado del filtro de carteles: 6.5 milisegundos.

\item \textbf{Par�metros}: El m�dulo de generaci�n de par�metros tuvo algunos inconvenientes. En un principio, hicimos un programa con el c�digo base de lo que iba a ser el m�dulo, consistente en una ventana que generaba estructuras de datos con los valores elegidos. El funcionamiento del programa fue exitoso, cosa que vimos imprimiendo por pantalla el contenido de dichas estructuras. Cuando integramos el m�dulo (ya programado como tal) en la aplicaci�n, tuvimos algunos problemas, pues los mensajes no llegaban bien al m�dulo de filtros. Las pruebas nos llevaron a la conclusi�n de que fue un fallo de arquitectura, con lo que tuvimos que remodelar el dise�o del n�cleo del pipeline para que admitiese m�s tipos de conexiones. Tras esto, el resultado fue satisfactorio.

\begin{figure}[h]
  \centering
  \includegraphics[scale=0.4, bb=0 0 363 206]{parametros.png}
% robot1.png: 72.009dpi, width=5.68cm, height=4.27cm, bb=0 0 161 121
  \caption{Pruebas satisfactorias de los par�metros}
\end{figure}

\item \textbf{OCR}: Ha sido uno de los principales retos de este proyecto. Al principio pensamos en utilizar un OCR ya implementado, pero ten�a el problema de que no se ajustaba completamente a nuestras necesidades. Probamos a implementarlo nosotros con algoritmos ya realizados, como descriptores de regiones, pero esos sistemas resultaban ser demasiado lentos cuando el numero de caracteres a reconocer dentro de la imagen aumentaba y como siempre el tiempo ha sido un factor que ha corrido en nuestra contra. Llegamos a plantearnos usar la red neurnal tambien en este campo, pero comprobamos que su entrenamiento era muy costoso, adem�s que no cumplia con las exigencias como por ejemplo que el tama�o de los caracteres no importe. Desarrollamos por tanto nuestro propio algoritmo descriptor reconocedor de caracteres basado en descripci�n de fronteras. La cantidad de informaci�n que este m�todo necesita para describir la frontera de un objeto es muy peque�a lo que acelero todo el proceso. Si queremos que sea muy preciso y que nunca se equivoque reconociendo un objeto aumentamos la cantidad de informaci�n que debe tratar, esto aumenta el tiempo ligeramente, as� que en vez de hacer eso decidimos usar a la salida del OCR un diccionario que comprobara si las palabras que sal�an ten�an o no sentido, si no lo ten�an las correg�a.
\item \textbf{Gesti�n de mensajes}: La gesti�n de mensajes fue probada a trav�s de su funcionamiento, y mostrando la salida por consola. La comprobaci�n de la correcci�n la hemos realizado imprimiendo por la salida est�ndar el estado del grafo de mensajes en todo momento.
\item \textbf{Post-gesti�n de �rdenes}: El m�dulo de gesti�n total de la tuber�a de �rdenes ha tenido pruebas triviales, debido a su sencillez, simplemente, hemos certificado mediante el uso que las �rdenes llegaban bien a los m�dulos de salida.
\item \textbf{Proceso de texto}: El proceso de texto ha sido implementado en Prolog, por lo que las pruebas han sido realizadas de una manera externa a la aplicaci�n, con el int�rprete de SWI-Prolog. Gracias a este m�todo, el desarrollo fue m�s r�pido, ya que la comprobaci�n conun int�rprete es muy �gil. Cuando funcion� por separado, la integraci�n en la aplicaci�n principal no caus� ning�n problema, y funcion� tal y como lo hab�amos previsto, con lo que no hizo falta realizar m�s pruebas que la pura comprobaci�n del funcionamiento en ejecuciones normales. Podemos ver un ejemplo en \ref{pruebas_proceso}.

\begin{figure}[h]
  \centering
  \includegraphics[scale=0.5, bb=0 0 110 35]{pruebas_texto.png}
% robot1.png: 72.009dpi, width=5.68cm, height=4.27cm, bb=0 0 161 121
  \label{pruebas_proceso}
  \caption{Proceso de texto}
\end{figure}

\item \textbf{Robot}: El desarrollo del robot tuvo un trabajo paralelo. Al principio, las pruebas fueron paralelas a la construcci�n de la estructura, y control�bamos los motores mediante un control de corriente cont�nua. Las pruebas nos llevaron a la conclusi�n de que hab�a que remodelar los �rboles de engranajes y los cambios de par, pues en un principio no suministraban la suficiente potencia como para mover todo el peso de la estructura. Tuvimos que cambiar esto, y fue un cambio bastante grande. Una vez que el robot se mov�a con el control remoto, probamos a crear el circuito que iba a hacer de capa entre el puerto paralelo y los motores. Conectamos el circuito, y sobre la placa las tensiones funcionaban bien. Lo ensamblamos al robot, y, aunque en un principio funcionaba bien, pronto dej� de hacerlo. Tras depurar, vimos que hab�a un fallo en un punto de la placa (nos cost� bastante averiguarlo), y, una vez corregido esto, el robot comenz� a funcionar de una manera muy estable.
\item \textbf{Red neuronal}: Las pruebas sobre la red fueron realizadas antes de empezar el proyecto. Se realizaron pruebas sobre cientos de fotos para el reconocimiento de ciertos patrones, para ello implementamos un peque�o programa de entrenamiento donde contabilizamos el �ndice de aprendizaje sobre el conjunto de fotos de entrenamiento, el �ndice de fallos que comet�a sobre otro conjunto de prueba y sobre otro de validaci�n. Se contabilizaba tanto el �ndice de aciertos como de error, modificando el factor de aprendizaje, el conjunto de entrenamiento y el numero de iteraciones.

\bigskip
Concretamente para el reconocimiento de los 4 gestos que se hacen con la mano al robot se necesitaron 148 fotos de entrenamiento, 49 fotos de validaci�n, 30 de prueba y 20 vueltas de aprendizaje, esto equivale en un Pentium4 a 3 horas de aprendizaje ajustando los pesos de la red.

Porcentaje de aciertos en entrenamiento: 89   Error medio: 0,0141046521582562

Porcentaje de aciertos en validaci�n: 93    Error medio: 0,00799933215976971

Porcentaje de aciertos en prueba: 100     Error medio: 0,00645778571591585v

\begin{figure}[h]
  \centering
  \includegraphics[scale=0.4,bb=0 0 504 231]{Grafica_Aciertos.png}
  \includegraphics[scale=0.4,bb=0 0 491 237]{Grafica_Errores.png}
  \caption{ La imagen de la izquierda muestra como aumenta el aprendizaje cuanto mas se ense�a a la red. Aumentando el n� de aciertos. Y la de la derecha muestra como a medida que aprende comete menos errores reconociendo figuras.}
\end{figure}

Una vez guardados los pesos en un archivo solo hay que cargarlos en una nueva red cada vez que se quieran utilizar y la retropropagaci�n para el reconocimiento de futuros patrones es casi instant�nea, que es lo que realmente importa. La eficiencia de la red reconociendo patrones es mejorada por el uso de los filtros previos.

\item \textbf{Entorno 3D}: Inicialmente la simulaci�n 3D se prob� exahustivamente de forma aislada para cetificar el correcto funcionamiento de cada uno de los elementos especificados(movimiento del robot, seguimiento de la c�mara, iluminaci�n, etc). En dichas pruebas se utiliz� una entrada directa por teclado gestionada internamente por la aplicaci�n para controlar la navegaci�n del robot a trav�s del escenario. A la hora de realizar la integraci�n con el resto de m�dulos, simplemente se sustituy� el control por teclado interno por los comandos recibidos a trav�s de los puertos de conexi�n resultantes de los an�lisis previos de los gestos del usuario. Una vez establecida la conexi�n del m�dulo con el pipe, s�lo hubo que probar que cada uno de los comandos analizados produc�an el comportamiento deseado en la simulaci�n.Adem�s, como elemento redundante se a�adi� una salida en texto que mostrara al usuario el comando en ejecuci�n en un instante determinado.
\begin{figure}[h]
  \centering
  % No se si los par�metros son correctos para la imagen.(lordarkam)
  \includegraphics[scale=0.5, bb=0 0 202 151]{templo1.png}
  \caption{Entorno 3D}
\end{figure}
\item \textbf{Salida de texto}: Para las pruebas de la salida de texto simplemente hemos comprobado que el texto que mand�bamos al m�dulo sal�a por la ventana, a�adimos un ejemplo en la figura \ref{pruebas_salida}.

\begin{figure}[h]
  \centering
  \includegraphics[scale=0.5, bb=0 0 80 35]{salida_texto.png}
% robot1.png: 72.009dpi, width=5.68cm, height=4.27cm, bb=0 0 161 121
  \label{pruebas_salida}
  \caption{Salida de texto}
\end{figure}

\item \textbf{Joystick}: La comprobaci�n del m�dulo de joystick se ha ido haciendo a medida que la integraci�n avanzaba. Al ser un m�dulo que se ha desarrollado en la fase final de la aplicaci�n, la estructura de la misma ya estaba bastante s�lida, y el funcionamiento del m�dulo ha sido casi inmediato.
\end{itemize}


\part{Detalles de los m�dulos}
\label{anexo1}
\chapter{M�dulo de generaci�n de im�genes}

\section{Introducci�n}
Este m�dulo tiene como objetivo la creaci�n y modificaci�n de un b�fer de colores (una imagen en formato plano) para la alimentaci�n de la tuber�a de visi�n. Obtiene, de diferentes fuentes, imagenes codificadas de diferentes maneras, y genera, en un formato unificado, una imagen sin comprimir.

\section {Im�genes generadas}
\label{formato_imagenes}
La funcionalidad principal de este m�dulo es, como hemos comentado antes, la de generar im�genes de un solo formato, independientemente del origen del que se obtengan. Con este procedimiento pretendemos establecer la base del �rbol de m�dulos de nuestro proyecto. Previamente al dise�o y la implementaci�n del m�dulo, establecimos cu�les eran los requisitos que deb�a cumplir el formato de im�genes que �bamos a usar. En primer lugar, las im�genes deb�an ser en color. Para esto, deb�amos ser capaces de manejar la profundidad de cada punto de la pantalla. Adem�s de esto, quer�amos que las dimensiones de las im�genes (ancho y alto) fuesen flexibles, y, a priori, ser capaces de manejar cualquier tama�o (aunque posteriormente esto no ha sido realmente posible, pues la c�mara web s�lo admite un conjunto de dimensiones determinado).

Estos dos requisitos, m�s la tendencia que hemos intentado mantener de mantener el proyecto lo m�s simple posible (filosof�a {\em KISS}\footnote{Keep it simple, stupid!}), nos han llevado a definir nuestro formato de im�genes interno de la siguiente forma:

\begin{itemize}
  \item[Alto] Un entero que determina el alto de la imagen (ej.: 240).
  \item[Ancho] Un entero que determina el ancho de la imagen (ej.: 320).
  \item[Bytes] Un entero que determina el n�mero de bytes por punto (ej.: 3).
  \item[Buffer] Un puntero (en la implementaci�n de C, realmente implementa un vector) de n�mero de 8 bits sin signo (1 byte, corresponde al formato {\tt unsigned char} en C), que contiene la informaci�n de colores de la imagen.
\end{itemize}

Esta ha sido toda la informaci�n por imagen que nos ha sido necesaria. El hecho de disponer de un b�fer lineal nos ha capacitado para escribir implentaciones muy r�pidas del recorrido de las mismas, principalmente por la potencia de los punteros del lenguaje C.

\section {Arquitectura y funcionamiento del m�dulo}
El m�dulo sigue las interfaces de comunicaci�n con el {\em pipeline}, de modo que crea los b�feres en la funci�n de iniciar, a la vez que, seg�n se haya instanciado a trav�s de la configuraci�n del XML que define el proyecto, abre la comunicaci�n con las bibliotecas pertinentes, en funci�n de los argumentos.

En el ciclo de im�genes se procede seg�n sea el funcionamiento. En el caso de que las im�genes cambien cada ciclo (no pasa cuando son im�genes de un color fijo o cargadas de archivo), se obtiene del recurso indicado el b�fer en el formato que ofrezca la biblioteca, y se transforma al formato de salida com�n, como se ha comentado en la secci�n anterior. Una vez que se ha hecho esto, se ``deposita'' en el puerto de salida la imagen resultante, habiendo ya dado uniformidad a las diferentes entradas, haciendo que el {\em pipeline} no dependa de las fuentes generadores de im�genes a m�s bajo nivel.

En la funci�n de cerrar simplemente libera los recursos.

\section {Bibliotecas}
Para la generaci�n resultados desde diferentes fuentes, el m�dulo hace uso de una serie de librer�as; son las siguientes:

\begin {itemize}
\item {\bf DirectX}: Las bibliotecas {\em DirectX} nos han provisto de la interfaz necesaria en su m�dulo {\em DirectShow}, tanto para la adquisici�n de im�genes de c�mara % TODO: esto no se sabe :P % como para la reproducci�n de un archivo de v�deo.
\item {\bf SANE}: Sane (Scanner Access Now Easy) es una biblioteca originariamente para interfaces con esc�neres, pero ampliada para cualquier dispositivo de im�genes. A trav�s de esta biblioteca accedemos a la c�mara.
\item {\bf Xine}: Xine-lib tiene como objetivo la reproducci�n de archivos de v�deo en varios formatos (los m�s usuales, tambi�n puede aceptar formatos nuevos a trav�s de {\em plugins}). A trav�s de Xine cargamos un v�deo en el m�dulo de im�genes y lo reproducimos.
\item {\bf Gdk}: Usamos Gdk para cargar f�cilmente archivos de im�genes, y usarlos as� como fuentes sencillas de im�gnenes cuyo resultado se conoce.
\item {\bf C�digo propio}: Tambi�n hemos implementado, para pruebas principalmente, las siguientes funcionalidades del m�dulo:
  \begin {itemize}
  \item {\bf Colores planos}: Pintamos todo el b�fer de un color. Es muy �til para comprobar el funcionamiento de los filtros.
  \item {\bf Im�genes aleatorias}: Rellena la matriz de colores con colores al azar, de esta manera vemos c�mo los filtros aceptan o dejan de aceptar los valores.
  \end {itemize}
\end {itemize}

\section {C�digo}
Ver anexo: documentaci�n del m�dulo de imagenes.c, en \ref {imagenes_8c} (p�gina \pageref{imagenes_8c}).

\chapter{Red Neuronal}
\hline 

\section{Introducci�n}
La elecci�n de una red neuronal como medio de resoluci�n del problema de reconocimiento de patrones fue tomada debido a que es uno de los sistemas de aprendizaje autom�tico mas recomendado en estos casos. Es muy �til cuando no se conoce la funci�n objetivo y se estima que los datos de entrada llegan con cierto porcentaje de ruido. La decisi�n se tom� meses antes de empezar el proyecto. Para asegurarnos de que era viable implementamos la red dentro de un peque�o programa, que a partir de im�genes de entrada devolv�a cierto si detectaba una imagen de una persona con una guante blanco y falso si no hab�a guante.

Para saber m�s: introducci�n sobre redes neuronales, en \ref {Introduccion_redes} (p�gina \pageref{Introduccion_redes}).

\section{Descripci�n t�cnica}
La red neuronal utilizada es una red multicapa de sigmoides con retropropagaci�n. 

Para saber m�s: descripci�n t�cnica de nuestra red, en \ref {Descripcion_tecnica} (p�gina \pageref{Descripcion_tecnica}).

\section{Dise�o}
Esta formada por una capa de entrada, una de salida y una sola capa oculta.
Si imagin�semos la red como una caja negra, esta tendr�a que recibir como entrada una imagen y sacar como salida una cadena de texto explicativa de alg�n atributo de esa imagen. 

En nuestro caso la entrada ser�n siempre im�genes del mismo tama�o 320x240 p�xeles, por ello la capa de entrada consta de 76800 unidades. En realidad la entrada es un char* que representa la imagen con valores de 255 o 0, es decir, blanco o negro, recuerdo que las im�genes que le llegan a la red son im�genes que previamente han pasado por el modulo de filtro a si que llegan ya binarizadas. Estas entradas ser�n normalizadas entre 1 y 0, para que las entradas est�n en el mismo rango que las unidades de la capa oculta y de salida. 

La capa oculta debe tener tan pocas unidades como sea posible. Medidas experimentales demuestran que el hecho de aumentar el n�mero de  unidades ocultas proporciona mejoras poco significativas en la precisi�n, pero requieren mucho mas tiempo de entrenamiento. Nosotros hemos optado por utilizar 15 unidades.

Como ya sabemos al robot se le controla con 2 manos, los gestos de la mano izquierda le indican las ordenes y los de las derecha los par�metros, con el objetivo de simplificar el dise�o los gestos de ordenes y de par�metros son los mismos, pero significan cosas distintas. Tanto para ordenes como para par�metros hay 5 tipos de gestos. Como recordatorio las ordenes eran: parar, avanzar, girar a la izquierda y girar a la derecha. Y los par�metros eran nula, medio baja, medio alta, alta si la orden actual es la de avanzar, donde los par�metros indican la velocidad a la que debe hacerlo o 0�, 45�, 90�, 180� si la orden actual es una de giro. La 5� gesto tanto para ordenes como para par�metros es el ``gesto no reconocido''. Por tanto dado que hay 5 tipos de gestos ha reconocer, la red tendr� que sacar 5 posibles salidas. Al principio optamos por una capa de salida de una sola unidad. El valor oscila entre 0 y 1 as� que por ejemplo si esta unidad val�a 0.2 significaba que hab�a reconocido el 2� gesto, si val�a 0.8 hab�a reconocido el 4� gesto. Luego se cambio al dise�o actual que es una capa de salida de 5 unidades, esto hace a la red mucho mas fiable, se pod�a decir que la salida de la red antes era anal�gica y ahora es digital, ya que todas las salidas tendr�n valores menores de 0.5 excepto una que ser� mayor, solo hay que asociar la unidad de la salida que se ha puesto en alta con una cadena de texto. Esta asociaci�n se hace mediante un script en Lua, as� es m�s modificable ya que si se quiere cambiar el texto de salida no hay que recompilar el proyecto, solo cambiar un archivo de texto.

La organizaci�n de red por capas es est�ndar, la salida de cada unidad alimenta a todas las unidades de la siguiente capa.
La tasa de aprendizaje utilizada ha sido 0.3. La mas alta posible para reducir el tiempo de aprendizaje sin disminuir la precisi�n.
El descenso de gradiente es incremental para reducir el riesgo de quedarnos en m�nimos locales.
Los pesos de la unidades de salida y oculta son inicializados con peque�os valores aleatorios entre 0.05 y -0.05.

\section{Entrenamiento}
La mayor parte del c�digo utilizado en la red esta dirigido al entrenamiento. Por eso decidimos hacer un programa aparte que contiene el c�digo de entrenamiento de la red y luego el c�digo que est� presente en el proyecto que solo contiene el necesario para crear una red, calcular los valores de las capas a partir del valor de la capa de entrada y generar la cadena de texto de salida. As� el c�digo del proyecto queda mas sencillo para leer.

El proceso de entrenamiento empieza con la sesi�n fotogr�fica, es necesario hacer mas de un centenar de fotos para obtener un entrenamiento medianamente fiable. Nosotros para el entrenamiento de ordenes sacamos 185 fotos, consiste en sacar fotos d�ndole ordenes al robot correctas, err�neas o simplemente no d�ndoselas. Todas estas fotos han de ser filtradas del mismo modo que lo har�a el modulo de filtro del proyecto, la raz�n de hacer un filtrado previo es poder pasar a la red im�genes muy simples, tambi�n deben de ser tomadas en unas condiciones de iluminaci�n similares a las que tendr� el entorno por el que circule el robot. No es lo mismo hacer aprender a la red a reconocer un gesto perdido en un mar de p�xeles de miles de colores a reconocer un conjunto de p�xeles blancos centrados sobre un fondo negro. Los objetivos mas perseguidos en este proyecto es la eficiencia y en este caso la fiabilidad en el reconocimiento. 

Las fotos son nombradas con un formato determinado, por ejemplo, ``\_orden\_parada\_21.bmp'' esto significa que la foto contiene la orden de parada y ``\_no\_gesto\_51.bmp'' indica que la foto no representa ninguna orden para el robot. Este formato es utilizado en el entrenamiento para que la red sepa ir reajustando los pesos seg�n el nombre explicativo de la foto.

%\begin{center}
\includegraphics{\_no\_gesto\_36.eps} %[scale=1]
%  \_no\_gesto\_36.bmp
%\end{center}

Todas las fotos no son utilizadas para el entrenamiento. Se hacen 3 listas de fotos que se utilizaran para el entrenamiento, la prueba y la validaci�n. Estas 2 ultimas sirven  para comprobar el buen funcionamiento de la red entrenada.
El objetivo del programa de entrenamiento es crear una red, entrenarla y salvar la estructura y pesos de red en un archivo. 
El entrenamiento consiste en :
\begin{itemize}
\item Recorrer la lista de im�genes de entrenamiento una por una.
\item Cargar la imagen en la imagen en la capa de entrada, cada valor de p�xel se asocia a una unidad de la capa.
\item Seg�n el nombre de la foto, ejemplo ``\_orden\_parada\_21.bmp'', se cambia el objetivo, esto sirve para calcular el error cometido.
\item Cambiado el objetivo, se calcula el valor de la capas respecto a la capa de entrada, se calcula el error cometido en las capas oculta y salida y se reajustan los pesos, para disminuir el error.
\item Esta lista es recorrida un numero finito de iteraciones. Las condiciones de parada pueden ser varias. La nuestra es simplemente un numero concreto, en este caso fueron 30 iteraciones. Por tanto los pesos fueron ajustados 30x(numero de fotos de la lista) veces.
\end{itemize} 	
Ya tendr�amos as� unos pesos que representan una aproximaci�n a la funci�n buscada.

\begin{center}
%  \includegraphics[scale=1]{prog_train.png} 
  Captura del programa implementado para el entrenamiento de redes.
\end{center}

Estos fueron datos de un entrenamiento de la red:
Datos entrada: 
	148 fotos de entrenamiento, 49 fotos de validaci�n y 30 de prueba.
	�ndice de aprendizaje: 0.3
	20 iteraciones
Datos salida:
	Porcentaje de aciertos en entrenamiento: 89 	Error medio: 0,0141046521582562
	Porcentaje de aciertos en validaci�n: 93 		Error medio: 0,00799933215976971
	Porcentaje de aciertos en prueba: 100 		Error medio: 0,00645778571591585

\begin{center}
%  \includegraphics[scale=1]{Grafica_Aciertos.png}
  Gr�fica de evoluci�n del porcentaje de aciertos respecto al numero de iteraciones. Sobre las fotos de entrenamiento.
  
%  \includegraphics[scale=1]{Grafica_Errores.png} .
  Gr�fica de evoluci�n del error medio respecto al numero de iteraciones. Sobre las fotos de entrenamiento.
\end{center}

\section{Red Neuronal en el proyecto}
Como cada modulo del pipeline, el modulo de red tiene un peque�o numero de funciones fijas utilizadas para ser llamadas desde el pipeline. Tres de ellas son ``red\_iniciar'', ``red\_cerrar'' y ``red\_ciclo''. Iniciar crea la red y carga el archivo creado por el programa de entrenamiento, por ejemplo el ``orden\_net''. La funci�n cerrar libera toda la memoria. Y la funci�n ciclo lo �nico que hace es recibir un char* que representa la imagen, cargar estos valores normalizados en la capa de entrada y calcular el valor de las capas oculta y de salida seg�n los pesos, que solamente tarda aproximadamente 0.10 segundos. Luego como ya dijimos solamente una de las cinco unidades de la capa de salida tendr� un valor superior a 0.5, esto es equivalente a que se ha puesto en ALTA y el modulo sacar� como salida la cadena de texto asociada a esa unidad. Cadena modificable desde un script junto con el nombre del archivo de la red entrenada. Todo lo que sea modificable en un futuro por posibles mejoras son par�metros que van escritos en scripts. 

Documentaci�n del c�digo de la red: en \ref {Codig_red} (p�gina \pageref{Codig_red}).

Estas son 5 im�genes filtradas de ejemplo, cada una representa una orden o un par�metro.
Recordatorio:
		1 dedo--> 	Orden:  Avanzar		 	Par�metro:  Medio baja o 45�
		2 dedos-->	Orden:	Girar derecha		Par�metro:  Medio alta o 90�
		3 dedos-->	Orden:  Girar Izquierda		Par�metro:  Alta o 180�
		5 dedos--> 	Orden:  Parar			Par�metro:  Nula o 0�

\begin{center}
 % \includegraphics[scale=1]{_orden_parada_17.png} 
  Salida: Si red de ordenes: Parar	Si red de par�metros: Nula o 0�
%  \includegraphics[scale=1]{_orden_avanza_38.png} 
  Salida: Si red de ordenes: Avanzar	Si red de par�metros: Medio baja o 45�
%  \includegraphics[scale=1]{_orden_angulo_43.png} 
  Salida: Si red de ordenes: Girar derecha	Si red de par�metros: Medio alta o 90�
%  \includegraphics[scale=1]{_orden_negAngulo_84.png} 
  Salida: Si red de ordenes: Girar izquierda	Si red de par�metros: Alta o 180�
%  \includegraphics[scale=1]{_no_gesto_36.png} 
  Salida: Si red de ordenes y par�metros: No gesto
\end{center}

\section{Evoluci�n}
El primer sistema utilizado para resolver el problema de reconocimiento de gestos, fue la implementaci�n de dos redes distintas una para ordenes y otra para par�metros, debido a que su estructura era distinta. 
Se fueron modificando los filtros con el objetivo de facilitar el aprendizaje a la red.
La segunda elecci�n fue utilizar el mismo numero de gestos tanto en ordenes como en par�metros as� se pudo conseguir la misma implementaci�n para ambas.
Luego se redujo considerablemente el c�digo, eliminando la parte de entrenamiento del proyecto. El c�digo de entrenamiento pasar�a a ser un programa a parte que generar� archivos de redes entrenadas. En este punto hab�a 2 archivos uno para cada red.
Y por ultimo visto que los gestos de las ordenes eran reconocidos con mucha mas facilidad que los asignados a los par�metros, que fallaban constantemente, decidimos que los gestos de los par�metros fuesen iguales a los de las ordenes, solo que en vez de hacer gestos con la izquierda se hacen con la derecha. De esta manera ambas redes cargan el mismo archivo y son igual de fiables. Solo se diferencian por la cadena de texto que devuelven, pero eso va por scripts.

\chapter{M�dulo de Entorno 3D}

\section{Introducci�n}

El entorno virtual 3d constituye un interface visual con el usuario que puede observar la simulaci�n de las evoluciones de un robot que se desplaza por un escenario siguiendo las �rdenes procesadas por el sistema de visi�n por computador.

Con esta aplicaci�n, es posible evaluar y testar las funcionalidades que se han implementado sin tener que llegar a la integraci�n del sistema en una entidad rob�tica real.

\section{Detalles}
\begin{itemize}
\item {\bf Entrada}: Una estructura de datos que contiene una cadena de �rdenes y otra de par�metros.
\item {\bf Salida}: Movimiento del robot virtual.
\item {\bf Descripci�n}: M�dulo que a partir de unas �rdenes representa el desplazamiento de un robot virtual a trav�s de un escenario en tres dimensiones.
\end{itemize}


\section{Especificaci�n}

Para cumplir con su cometido, la aplicaci�n debe poseer las siguientes caracter�sticas funcionales:

\subsection{Representaci�n tridimensional de un entorno/escenario}

Teniendo en cuenta que una de las principales motivaciones para elaborar un sistema de �rdenes mediante visi�n artificial es la de implementar un dispositivo de navegaci�n que permita a un aut�mata desplazarse por su entorno, es fundamental simular gr�ficamente un escenario sobre el cu�l el robot virtual pueda desenvolverse. Esta representaci�n debe ser lo m�s inmersiva posible para que la precisi�n de la simulaci�n sea adecuada.
\subsection{Representaci�n de un robot capaz de desplazarse por su entorno}

Debe existir una entidad (en este caso un modelo 3D) dentro de la simulaci�n que represente la ubicaci�n, orientaci�n y desplazamientos resultantes de las distintas �rdenes procesadas por el sistema de control. Aunque en principio no es relevante, se ha intentando que esta entidad cumpla con ciertos criterios de dise�o que empaticen con los sistemas motrices m�s comunes en aut�matas (uso de ruedas, orugas, etc).

\subsection{Sistema de control}

Las diferentes �rdenes que son procesadas por el sistema de visi�n artificial, son analizadas para establecer las nuevas propiedades de posicionamiento, direcci�n, etc de la entidad que representa al robot. Esta interacci�n se realiza mediante el paso de mensajes distribuidos por el pipe que se encarga de la interconexi�n entre los diferentes m�dulos. Adem�s, es necesario que de forma aut�noma la aplicaci�n pueda procesar diferentes comandos emitidos por el usuario mediante el teclado y rat�n para poder controlar otros aspectos secundarios de la simulaci�n, como son el posicionamiento de la c�mara, activaci�n de diferentes efectos gr�ficos, etc.

\subsection{Sistema de c�maras interactivas}

Para poder visualizar de forma �ptima todos los componentes de la simulaci�n, es necesario poseer un sistema de c�maras que permita seguir c�modamente las evoluciones del robot por el escenario, as� como permitir al usuario adaptar la posici�n de las distintas c�maras para obtener el �ngulo de visi�n m�s relevante en cada momento.

\section{Implementaci�n}

La consecuci�n de las especificaciones previamente expuestas se ha conseguido mediante la implementaci�n de un amplio conjunto de caracter�sticas relacionadas principalmente con la programaci�n gr�fica 3D. A continuaci�n se describen detalladamente:

\subsection{Uso de DirectX 9.0}

La implementaci�n de los gr�ficos tridimensionales se ha realizada sobre la librer�a gr�fica Direct3D incluida en DirectX 9.0. Se ha optado por este est�ndar en vez del uso tambi�n muy extendido de la librer�a OpenGl por razones acad�micas, en el sentido de que previamente hab�amos tenido experiencia con OpenGl en otros proyectos y esta se presentaba como una oportunidad id�nea para investigar nuevas tecnolog�as. Cabe se�alar que ambas librer�as poseen similar potencialidad por lo que la elecci�n por razones funcionales no era especialmente relevante.
\subsection{Modelos 3D creados sobre 3dStudio Max 6.0}

Se ha empleado la herramienta de modelado 3dStudio Max para la elaboraci�n de los modelos tridimensionales tanto del escenario como del robot. Para su posterior integraci�n con la aplicaci�n, se ha utilizado un conversor que compatibiliza el formato usado por 3dStudio con el usado por Direct3D (archivos .X). En cuanto al apartado de las texturas, se ha utilizado Adobe PhotoShop para su realizaci�n.
\subsection{Terreno generado a partir de un mapa de altura}

Como parte del escenario, se ha incluido un terreno que es generado de forma procedural a partir de un mapa de altura. Una vez calculado la malla 3D de dicho terreno, se genera de forma procedural  la textura a partir de diferentes im�genes que se interpolan siguiendo como criterio la altitud del terreno. De esta forma, seg�n lo alto o bajo que est� el terreno, este presentar� el aspecto de diferentes materiales ( hierba para las zonas bajas, roca en las cumbres de las monta�as, etc). Por �ltimo se utiliza un algoritmo primitivo de sombreado que ilumina la textura resultante de forma coherente respecto a la posici�n de la luz en el escenario ( en este caso del sol).

\subsection{Iluminaci�n din�mica}

Para otorgar una mayor sensaci�n de integraci�n de la entidad que se desplaza con  el escenario, se ha utilizado iluminaci�n din�mica. De esta forma el robot es iluminado correctamente seg�n su posici�n actual, incrementando adem�s el realismo en la percepci�n de los materiales al visualizarse efectos de brillo, luz difusa, variaci�n crom�tica,etc.


\subsection{Iluminaci�n est�tica. LightMaps}

Los elementos est�ticos del escenario no requieren de iluminaci�n din�mica ya que su posici�n no var�a durante la ejecuci�n de la aplicaci�n. Para ahorrar recursos, es habitual el uso de texturas secundarias tambi�n llamadas lightmaps que codifican la informaci�n sobre la iluminaci�n que ese objeto recibe. En este caso, se ha utilizado una �nica textura resultado de la fusi�n de la textura primaria y los lightmaps para optimizar el rendimiento.


\subsection{Sombreado Din�mico. Stencil Buffer}

De forma an�loga a la iluminaci�n din�mica, el sombreado din�mico es un efecto que permite la integraci�n de objetos m�viles en escenarios de forma muy realista. Se ha implementado un algoritmo de sombreado que se basa en el uso del Stencil Buffer. Este algoritmo es bastante costoso en cuanto a recursos de la tarjeta gr�fica, por lo que sueles implementarse para ser soportado por tarjetas con aceleraci�n 3D de �ltima generaci�n.

\subsection{Luces Glow. Lens Flares}

Las luces Glow son puntos lum�nicos que producen un haz a su alrededor cuando se mira directamente. En este caso, se ha implementado un punto de luz que representa al sol. Adem�s, se ha implementado un efecto conocido como Lens Flares que se produce cuando se enfoca un sistema �ptico (c�mara de fotos, de video,etc) sobre un foco de luz intensa. Este efecto produce  una serie de reflejos residuales que se ubican de forma relativa a la luz que los origina. Se ha introducido por razones est�ticas y para incrementar la inmersi�n en el entorno virtual.

\subsection{Sistema de c�maras}

Se han implementando varios sistemas de c�maras para ofrecer m�ltiples posibilidades al usuario de seguir la acci�n que se desarrolla en la simulaci�n. Se dividen en:

\begin{itemize}

\item C�mara de seguimiento r�gido: La c�mara se sit�a siempre detr�s del robot y sigue cada uno de sus movimientos permaneciendo siempre a una misma distancia.

\item C�mara de seguimiento orbital: A diferencia de la anterior, esta c�mara orbita alrededor del robot manteniendo siempre su punto de enfoque fijado en �l.

\item C�mara fija: Se establece una posici�n de la c�mara donde permanece fija mientras enfoca y sigue los movimientos del robot.

\item C�mara libre: Con esta c�mara se puede navegar por todo el escenario as� como enfocar a los puntos de mayor inter�s para el usuario.
\end {itemize}

\subsection{Sistemas de part�culas}

De forma gen�rica, se ha implementado un sistema de part�culas para poder introducir efectos especiales y atmosf�ricos como pueden ser humo, fuego, lluvia, nieve, etc. Los sistemas de part�culas combinan una representaci�n gr�fica mediante billboards( pol�gonos especiales que mantienen siempre su orientaci�n respecto a la c�mara) y un sistema de control f�sico que determina el comportamiento de las part�culas en cuanto a aceleraci�n, direcci�n, tiempo de vida, turbulencia, etc.


\section{Dise�o}

El dise�o de la aplicaci�n es bastante sencillo en cuanto a su estructura jer�rquica. Existe una clase que contiene a la VentanaPrincipal de la aplicaci�n y que constituye el n�cleo de la ejecuci�n. Esta clase contiene un objeto del tipo Escena que es el que administra el resto de entidades con sus correspondientes representaciones gr�ficas. As�, la escena es el contenedor para otros objetos como son el Terreno, el Cielo, SistemaPart�culas, C�mara, y el resto de entidades como el escenario y el propio robot que pertenecen a la clase Objeto. Respecto a esta �ltima clase, destacar que posee como atributo un objeto de la clase Mesh que constituye la representaci�n gr�fica de la entidad (es decir, la maya 3d). Para gestionar la carga de los diferentes modelos 3d, existe una clase est�tica llamada MeshManager que comprueba que no se carguen en memoria varios modelos del mismo tipo. Por tanto, la creaci�n de objetos del tipo Mesh se realiza siempre a trav�s de esta clase gestora y nunca directamente. Por �ltimo, existen dos clases est�ticas que se encargan de la interacci�n del usuario mediante el rat�n y teclado ( clase Entrada) y otra que gestiona la impresi�n de texto en pantalla (clase Texto).

Respecto al flujo de ejecuci�n, se puede resumir en que existe un m�todo que es llamado externamente de forma peri�dica. Dicho m�todo hace que en cada ciclo se actualice la posici�n de cada uno de los objetos de la escena y posteriormente se renderice. Este proceso se hace de forma delegada, de forma que VentanaPrincipal llamar� al m�todo Render() de la Escena, la cu�l llamar� recursivamentea a los respectivos m�todos Render() de cada una de las entidades que contiene.

\section{Capturas}

% ,bb=0 0 512 384
%,bb=0 0 512 384
% ,bb=0 0 512 384

Se muestran a continuaci�n una serie de capturas de pantalla del m�dulo en funcionamiento:
\begin{figure}[h]
  \centering
  \includegraphics[scale=1.0 ,bb=0 0 512 384]{templo1.png}
  \caption{Primer plano del robot.}
\end{figure}

\begin{figure}[h]
  \centering
  \includegraphics[scale=1.0 ,bb=0 0 512 384]{templo2.png}
  \caption{Seguimiento del robot con una c�mara est�tica.}
\end{figure}

\begin{figure}[h]
  \centering
  \includegraphics[scale=1.0 ,bb=0 0 512 384]{templo3.png}
  \caption{Visi�n general del escenario.}
\end{figure}

%\chapter{Filtros}

\hline
\linebreak 

La visi�n artificial es el proceso sensorial m�s complejo de todos.
Las tareas en vis�n por computador se pueden enumerar en:
\begin{enumerate}
\item Visi�n de bajo nivel. (Tareas autom�ticas)
  \begin{itemize}
  \item Captaci�n. Obtenci�n de la imagen.
  \item Preprocesamiento. Incluye t�cnicas como reducci�n de ruido y realce de detalles.
  \end{itemize} 
\item Visi�n de nivel medio. (Etiquetar objetos)
  \begin{itemize}
  \item  Segmentaci�n. Localizaci�n de los objetos de inter�s.
  \item Descripci�n. Obtenci�n de caracter�sticas: tama�o, formas, etc.
  \end{itemize} 
\item Visi�n de alto nivel. (Emular inteligencia)
  \begin{itemize}
  \item Reconocimiento. Identificaci�n de objetos: tornillos, puertas, etc.
  \item Interpretaci�n. Significado de un conjunto de objetos.
  \end{itemize} 
\end{enumerate} 

El objetivo de los filtros utilizados en nuestro proyecto entra en el �mbito del preprocesamiento y segmentaci�n de im�genes.
Ejemplos de la fase de preprocesamiento son el suavizado, el realce, detecci�n de bordes, detecci�n de umbral, etc.
El procesamiento de una imagen puede ser visto como una transformaci�n de una imagen en otra imagen, es decir, a partir de una imagen, se obtiene otra imagen modificada. Desde el punto de vista de visi�n artificial, el �nico prop�sito del procesamiento de im�genes es conseguir mas adelante un an�lisis de estas mas simple y mas fiable. Por consiguiente, el procesamiento de im�genes debe facilitar la extracci�n de informaci�n para un posterior an�lisis, de manera que la escena pueda ser interpretada de alguna manera.

Por este motivo aplicamos a la imagen capturada por la webcam una serie de filtros, para simplificar la imagen, hasta el punto de eliminar la informaci�n que no nos interesa y realzar la informaci�n importante para el an�lisis posterior de la imagen, en este caso el reconocimiento de gestos y carteles.

Para saber mas sobre im�genes digitales: \ref {Imagen Digital} (p�gina \pageref{Imagen Digital}).

La capacidad visual del robot, depender� de los m�dulos de visi�n que est�n activos. De momento solo hay m�dulos implementados y activos que permiten al robot recibir ordenes con gestos hechos con unos guantes o tambi�n recibir informaci�n procedente de carteles. Los carteles no solo pueden darle ordenes, si no hacerle una pregunta de la cual tenga conocimiento o hacerle realizar una operaci�n aritm�tico-l�gica.

Por tanto los �nicos medios para comunicarse con el robot son guantes y carteles especiales, de momento.

Son especiales por su color. La raz�n de utilizar colores especiales ha sido crear en las im�genes capturadas, regiones de color con unos rangos de intensidades en los tres colores, mas separadas del resto de intensidades del histograma de la imagen. As� podremos aislar esta regi�n, la del color especial. En el caso de los guantes, es la posici�n de los dedos la que indica la orden, es en los dedos donde esta el color especial, as� que si solo nos quedamos con las regiones de este color y el resto lo despreciamos, estaremos simplificando much�simo la imagen para un posterior an�lisis de esta. Lo mismo pasar�a con los carteles, desechamos toda la imagen que no forme parte del cartel y dentro del cartel nos quedamos solo con la frase.

Filtro de guantes. \ref {Filtro de Gestos} (p�gina \pageref{Filtro de Gestos}).
Filtro de carteles. \ref {Filtro de Carteles} (p�gina \pageref{Filtro de Carteles}).

Los m�dulos posteriores a los filtros son m�dulos de an�lisis que deben de recibir la informaci�n lo mas clara posible, en el caso de los gestos se utiliza una red neuronal, la cual tiene que ser entrenada con im�genes muy simplificadas para que el entrenamiento tenga efecto y que las im�genes que reciba una vez entrenada, sean filtradas de la misma manera, para generar im�genes iguales que con las que fue entrenada, para poder reconocerlas. Respecto a los carteles el siguiente modulo es un OCR, muy sensible a ruidos, por tanto hay que asegurar que el filtro es efectivo, para que la salida de este no sea incoherente.


%\section{Imagen Digital}
\label{imagen_digital} 

Existen dos tipos de im�genes utilizadas frecuentemente en visi�n por computador: Im�genes de intensidad e im�genes de alcance.
\bigskip 

Las im�genes de alcance, tambi�n denominadas im�genes o mapas de profundidad, tiene su fundamento en los sensores de alcance �pticos y estiman directamente la estructura de la imagen en tres dimensiones de la escena.

Las im�genes de intensidad miden la cantidad de luz que incide en un dispositivo fotosensible. Es este tipo de im�genes sobre las cuales hemos trabajado en nuestro proyecto.
\bigskip 

Las estructuras de datos utilizadas en la digitalizaci�n de una imagen son varias:
\begin{itemize}
\item Tradicionales
\begin{enumerate}
\item Matrices
\item Cadenas
\item Estructuras topol�gicas
\item Estructuras relacionales
\end{enumerate} 
\item Jer�rquicas
\begin{enumerate}
\item Pir�mides
\item Quadtrees
\end{enumerate} 
\end{itemize} 

Las que nosotros hemos utilizado han sido matrices, es decir, una imagen digitalizada formar�a una matriz tal que:

$ f(x,y) \equiv $ valor proporcional a la intensidad de luz en el punto (x,y)

\begin{figure}[h]
  \centering
  \includegraphics[scale=0.5,bb=0 0 359 186]{imDig.png}
  \caption{ Matriz de intensidades de una imagen.}
\end{figure}
	
Cada elemento de la matriz o p�xel tendr� un valor asignado que se corresponde con el nivel de luminosidad del punto correspondiente en la escena captada, dicho valor es el resultado de la cuantizaci�n de intensidad o nivel de gris.
\bigskip 

Se suele utilizar el termino bitmap para hacer referencia a un mapa de p�xeles, aunque en algunos ambientes se preserva para aquellos mapas de p�xeles en los que un p�xel se representa por un simple bit.
\bigskip 

Si la imagen es en blanco y negro, se almacena solo un valor por p�xel, en nuestro caso las im�genes son a color, lo que hace que los elementos de la matriz ya no tengan asociado un solo valor sino tres, correspondientes el rojo, verde y azul. 

A este modelo se le llama modelo de color RGB.

Cada valor de p�xel es un triple ordenado (r, g, b), donde r, g, b representan las intensidades de rojo ,verde y azul respectivamente. r, g, b son cadenas de bits, en nuestro caso, son de 8 bits, esto significa que el rango de valores var�a de 0 a 255, donde el 0 representa el negro absoluto y el 255 el blanco absoluto.
\bigskip 

Algunas correspondencias de colores en el modelo RGB son:
\begin{itemize}
\item Negro:	(0,0,0)
\item Azul:	(0,0,255)
\item Verde:	(0,255,0)
\item Rojo:	(255,0,0)
\item Amarillo:	(255,255,0)
\item Blanco:	(255,255,255)
\end{itemize} 
	
La profundidad del color se define como la suma de los bits asociados a los componentes r, g, b. Por ejemplo, si utilizamos 2 bits para el rojo, 2 para el verde y 2 para el azul, tendremos una profundidad de color de 6 bits por p�xel, y una paleta de colores de $ 2^{6} = 64 $ posibles colores.

La profundidad de color ideal es la de 24 bits por p�xel, conocida como color verdadero. Por encima de esta profundidad el ojo humano es incapaz de notar la diferencia.
\bigskip 

Dos factores que afectan a la calidad de una imagen es su resoluci�n espacial y en amplitud. Dependiendo del numero de pixels que tenga el dispositivo la imagen poseer� mas o menos resoluci�n espacial. Y la amplitud se refiere al rango de bits destinados para representar la intensidad de cada p�xel. 

A mas resoluci�n en ambos campos mas memoria se requiere, pero mas n�tida es la imagen. Un ejemplo de un gr�fico a color verdadero con una resoluci�n de 1280 x 1024, necesita:

\begin{center}
1280 x 1024 x 24 = 31457280 bits (casi 4 Mb)
\end{center} 
 	 
\begin{figure}[h]
  \centering
  \includegraphics[scale=0.5,bb=0 0 240 160]{arbol1.png}
  \caption{ Resoluci�n de 240x160 y 256 niveles de color.}
\end{figure}

\begin{figure}[h]
  \centering
  \includegraphics[scale=0.5,bb=0 0 235 157]{arbol2.png}
  \caption{ Resoluci�n de 48x32 y 256 niveles de color.}
\end{figure}

\begin{figure}[h]
  \centering
  \includegraphics[scale=0.5,bb=0 0 240 160]{arbol3.png}
  \caption{ Resoluci�n de 240x160 y 8 niveles de color.}
\end{figure}	 

Resumiendo, el termino imagen se refiere a una funci�n de intensidad bidimensional $ f(x,y) $, donde x, y son las coordenadas espaciales y el valor de f en cualquier punto (x, y) es proporcional a la intensidad de color.

Una vez que se tiene definido el sistema de ejes cartesianos, ya es posible realizar cualquier tipo de operaci�n. Estas operaciones ser�n los filtros, que aplicados sobre la imagen la modificaran. 


%\chapter{Filtro de Gestos}

\hline
\linebreak 

\section{Introducci�n}
Como ya hemos dicho este filtro sirve como preprocesamiento y segmentaci�n de la imagen. Ya que el suavizado reduce los ruidos y la extracci�n de regiones de color localiza los objetos de inter�s, con el objetivo de pasarle una informaci�n mucho mas comprensible y simplificada a la red neuronal.

Los gestos al robot se realizan con la ayuda de 2 guantes, uno para la mano izquierda que dar� las ordenes y otro para la derecha que indicar� los par�metros para dichas ordenes.
Cada guante tiene en la punta de los dedos unos marcadores de color especial, un color que no se encuentre formando parte del entorno (colores muy llamativos con una textura que no gener� brillos o sombras). 

Se determinaron una conjunto de ordenes, las justas para que un objeto pueda describir cualquier trayectoria sobre una superficie plana. Concluimos que estas podr�an ser �nicamente: avanzar y girar. Es necesario decirle la distancia que tiene que avanzar en cada momento, pero como eso no era simple, se introdujo la orden parar, as� mientras se mueve el robot tu decides cuando ha recorrido la distancia oportuna y detenerlo con una orden. Tambi�n se distingui� en la orden girar, entre girar a la izquierda y girar a la derecha. Con esto tenemos 4 tipos de ordenes distintas para dar al robot. Pero aun el robot necesita mas informaci�n sobre estas ordenes, como por ejemplo en la orden de giro, con cuantos grados tiene que realizarlo o en la orden de avanzar a cuanta velocidad debe moverse. Siguiendo con el objetivo de la simplicidad en vez de a�adir mas gestos diferentes a la misma mano, se utilizo la otra mano, es decir, una mano indicar�a las ordenes al robot y otra los par�metros seg�n el tipo de orden. 
Los par�metros son o de velocidad o de grados de giro, la velocidad puede ser nula, medio baja, medio alta o alta y los �ngulos de giro pueden ser 0�, 45�, 90� o 180�, es decir, cuatro par�metros en ambos casos, se utilizan los mismos s�mbolos para velocidad como para giro, por tanto solo existen 4 gestos diferentes que se puedan dar con la mano derecha para expresar los par�metros al robot. 
La coincidencia del numero de gestos utilizados para ordenes y el numero de gestos utilizados para par�metros, simplificara la implementaci�n de la red. Y el hecho de usar los mismos gestos para representar las ordenes con la mano izquierda y los par�metros con la mano derecha, simplificara tambi�n el entrenamiento de la red.

Resumiendo:
\begin{itemize}
\item Mano Izquierda --> Ordenes
  \begin{enumerate}
  \item Parar
  \item Avanzar
  \item Girar Izquierda
  \item Girar Derecha
  \end{enumerate} 
\item Mano Derecha --> Par�metros
  \begin{enumerate}
  \item Si orden == Avanzar entonces
    \begin{itemize}
    \item Parar
    \item Medio baja
    \item Medio alta
    \item Alta
    \end{itemize} 
  \item Si (orden == Girar Izquierda) or (orden == Girar Derecha) entonces
    \begin{itemize}
    \item 0 �
    \item 45�
    \item 90�
    \item 180�
    \end{itemize} 
  \end{enumerate} 
\end{itemize} 					

Los gestos elegidos son diferentes posiciones de los marcadores del guante, lo mas claro posible, para que despu�s del filtrado la red no tenga problema para diferenciar unos s�mbolos de otros.
Los marcadores son las �nicas �reas de la imagen que no se eliminaran de la imagen. Estas regiones pasar�n a ser blancas y el resto negro. Por tanto repito los gestos tienen que ser los suficientemente distintos unos de otros, para que una vez filtrados, esas zonas blancas puedan diferenciarse a simple vista y saber a que orden se est�n refiriendo. 
Es en ejecuci�n cuando se decide a trav�s de una ventana proporcionada por el pipeline el color de la imagen a filtrar, as� que el color de los marcadores del guante no tienen porque ser fijos, se pueden determinar en cada momento. Eso si, los colores de ambos guantes deben de ser distintos y especificarse que color ser� el que representa a las ordenes y cual representara a los par�metros.

Estos son los 4 posibles gestos:
\begin{center}
\includegraphics[scale=1]{parar.png} 
Parada
\includegraphics[scale=1]{avanzar.png} 
Avanzar
\includegraphics[scale=1]{girarDcha.png} 
Girar Derecha
\includegraphics[scale=1]{girarIzq.png} 
Girar Izquierda
\end{center} 

Este ser�a el resultado de aplicar el modulo de filtro de gestos sobre esta imagen con una guante:
\begin{center}
\includegraphics[scale=1]{parar.png} 
\includegraphics[scale=1]{filtrado.png} 
\end{center} 

y esta sobre una imagen sin guante:
\begin{center}
\includegraphics[scale=1]{no_gesto.png} 
\includegraphics[scale=1]{filtrado_no_gesto.png} 
\end{center} 

Los filtros aplicados para esta simplificaci�n son:
\begin{itemize}
\item Suavizado. Por promediado del entorno. \ref {Suavizado de la imagen}
\item Extracci�n de regiones por el color. \ref {Extracci�n de regiones de color}
\item Centrado de imagen. \ref {Centrado de la imagen}
\end{itemize} 

\section{Suavizado de la imagen}
Difumina la imagen. Cuando no se aplica, la extracci�n de regiones posterior no es muy fiable, ya que a causa de la iluminaci�n o de la textura del material utilizado en el color especial, hay zonas del objeto de inter�s que no tienen el mismo color pudiendo pasar por ejemplo de ser un rojo casi blanco a un rojo casi negro, esta amplitud de color es inadmisible para la extracci�n de colores, ya que esos p�xeles ser�an considerados fuera de rango y por tanto como elementos del entorno y no como elementos de inter�s. Esto repercutir�a en la red neuronal posterior, la cual tiene que asignar pesos seg�n el valor de los p�xeles de entrada, si no podemos determinar unos valores fiables en las im�genes de entrada no se podr� entrenar de forma fiable la red ni poder asegurar un comportamiento seguro en el futuro.
El suavizado es una transformaci�n de vecindad, donde el valor del nuevo p�xel depende de los valores de los p�xeles que le rodean. Nuestro m�todo de filtro es un suavizado por el promediado del entorno de vecindad.
Para saber m�s sobre este filtro:  \ref {Suavizado} (p�gina \pageref{Suavizado}).
Gracias a esto, los p�xeles dentro de la regi�n de color de inter�s que sean ruidos generados por reflejos de luz o fallos de iluminaci�n, quedaran mas atenuados y todos los p�xeles dentro de la regi�n tendr�n mas posibilidades de estar dentro del rango de color especial buscado.
\begin{center}
\includegraphics[scale=1]{suavizado.png} 
\end{center} 

\section{Extracci�n de regiones de color}
Como hemos dicho los objetos de inter�s son las puntas de los dedos, as� que tenemos que aislarlas del resto de la fotograf�a. La forma es delimitar estas regiones y darlas toda la importancia respecto al resto de la imagen.
Queremos que el filtro convierta una imagen capturada por la webcam en una imagen blanca y negra, donde las puntas de los guantes quedaran en blanco y el resto de la imagen en negro. As� la red solamente ser� entrenada para recibir im�genes con regiones blancas y negras, si hay una sola regi�n de un cierto tama�o implicar�a que solo hemos ense�ado un dedo del guante, lo que se corresponder�a con el gesto de avanzar. 
Esto es a lo que llamamos segmentaci�n, ya que estamos localizando los objetos de inter�s.
Para saber mas sobre la extracci�n:  \ref {Extracci�n de regiones} (p�gina \pageref{Extracci�n de regiones}).
\begin{center}
\includegraphics[scale=1]{regiones.png} 
\end{center}  

\section{Centrado de la imagen}
Los gestos realizados con los guantes nunca son capturados por la webcam en la misma posici�n. Nunca estar�n totalmente centrados, si no ligeramente o totalmente desplazados mas a la derecho o mas a la izquierda, arriba o abajo. Esto es de vital importancia para la red, ya que entrena y reconoce en relaci�n a los valores de los p�xeles de la imagen, si aprende que el s�mbolo de avanzar es una regi�n blanca sobre un fondo negro situada siempre en el centro de la imagen, cuando este en fase de reconocimiento y la webcam capture un gesto desplazado, la red lo considerara como gesto no reconocido.
Otra opci�n podr�a ser entrenar la red para que reconociese el mismo gesto en cualquier posici�n, pero eso no podr�a nunca servir en el entrenamiento, y que los pesos no terminar�an nunca de fijarse, ya que el p�xel (x,y) si esta blanco para unas im�genes se considerara como ejemplo de entrenamiento positivo, para otras se considerara negativo y los pesos no podr�n ajustarse. Por tanto hay que intentar conseguir que las regiones de inter�s extra�das est�n siempre situadas mas o menos en la misma zona de la imagen. Para eso decidimos que la mejor forma de hacer esto era centrar la imagen seg�n el centro de masas del conjunto de p�xeles de inter�s.
As� siempre las regiones blancas que representan los gestos aparecen centrados en la imagen sobre un fondo negros, totalmente preparados para ser pasados a la entrada de la red neuronal.
Para saber mas sobre el centrado: \ref {Centrado} (p�gina \pageref{Centrado}).
\begin{center}
\includegraphics[scale=1]{filtrado.png} 
\end{center} 

\section{C�digo}
Documentaci�n del c�digo del filtro de gestos, en \ref {Codigo_filtro_gestos} (p�gina \pageref{Codigo_filtro_gestos}).


%\subsection{Suavizado}
\label{suavizado_label} 

Existen unos tipos de operaciones sobre im�genes como el suavizado y la extracci�n de bordes que reciben el nombre de transformaciones de vecindad.
\bigskip 

\subsubsection{Vecindad}
Se dice que todo p�xel p, de coordenadas (x, y) tiene 4 p�xeles que establecen con �l una relaci�n de vecindad horizontal y vertical que son:

\begin{center}
Horizontal: (x - 1, y),  (x + 1, y) 	
Vertical:  (x, y - 1),  (x, y + 1)
\end{center} 
\bigskip 

Estos cuatro p�xeles definen lo que se conoce como entorno de vecindad-4.

Por tanto vecinos de un p�xel (p) de coordenadas (x, y): (excepto bordes)
\begin{itemize}
\item $ N_{4}(p) $ = 4-vecinos de p {(x-1,y),(x+1,y),(x,y-1),(x,y+1)}
\item $ N_{d}(p) $ = vecinos diagonales de p {(x+1,y+1),(x-1,y+1),(x+1,y-1),(x-1,y-1)}
\item $ N_{8}(p) $ = 8-vecinos de p $ N_{4}(p) \cup N_{d}(4) $
\end{itemize} 

\begin{center}
\begin{tabular}{|c|c|c|}
\hline (x-1, y-1) & (x, y-1) & (x+1, y-1) \\ 
\hline (x-1, y) & (x, y) & (x+1, y) \\ 
\hline (x-1, y+1) & (x, y+1) & (x+1, y+1) \\ 
\hline 
\end{tabular} 

Entorno de vecindad del p�xel (x, y)
\end{center} 

\subsubsection{Conectividad}
Otra noci�n es la de conectividad. Conectividad para los p�xeles p y q con valores en V. Siendo V =  conjunto de valores de intensidad que se permiten que est�n adyacentes.
\begin{itemize}
\item 4-conectados		$ q \epsilon N_{4}(p) $
\item 8-conectados		$ q \epsilon N_{8}(p) $
\item m-conectados		$ q \epsilon N_{4}(p) $ o $ q \epsilon N_{4}(p) \cap {N_{4}(p) \cap N_{4}(q)=\phi} $
\end{itemize} 

Se dice que un p�xel q es contiguo a otro p�xel p si est�n conectados.

\subsubsection{Medidas}
Sea p = (x, y), q = (s, t), existen una serie de propiedades m�tricas.
\begin{itemize}
\item D(p, q) >= 0		D(p, q) = 0  si  p = q
\item D(p, q) = D(q, p) 
\item D(p, z) <= D(p, q) + D(q, z)
\end{itemize} 

$ D_{e} $ Distancia euclidea De(p, q) = $ \sqrt{(x-s)^{2} + (y-t)^{2}} $
\begin{figure}[h]
  \centering
  \includegraphics[scale=0.7,bb=0 0 84 98]{distEuclidea.png}
  \caption{ Distancia Euclidea.}
\end{figure}
 
$ D_{4} $ Distancia euclidea D4(p, q) = $ \vert x-s\vert + \vert y-t\vert $\  (Si $ D_{4}=1 \rightarrow 4-vecinos de p $)
\begin{figure}[h]
  \centering
  \includegraphics[scale=0.7,bb=0 0 84 98]{distEuclidea4.png}
  \caption{ Distancia Euclidea 4 vecinos.}
\end{figure}
 
$ D_{8} $ Distancia de ajedrez $ D8(p, q) = max(\vert x-s\vert , \vert y-t\vert)  $ (Si $ D_{8}=1 \rightarrow 8-vecinos de p $)
\begin{figure}[h]
  \centering
  \includegraphics[scale=0.7,bb=0 0 84 98]{distAjedrez.png}
  \caption{ Distancia de ajedrez.}
\end{figure}

Las t�cnicas mas utilizadas en el suavizado es el uso de m�scaras de convoluci�n (plantillas, ventanas o filtros). Normalmente son matrices 3x3, pero en nuestro caso hemos utilizado matrices 5x5.

\begin{center}
\begin{tabular}{|c|c|c|}
\hline w1(x-1, y-1) & w2(x, y-1) & w3(x+1, y-1) \\ 
\hline w4(x-1, y) & w5(x, y) & w6(x+1, y) \\ 
\hline w7(x-1, y+1) & w8(x, y+1) & w9(x+1, y+1) \\ 
\hline 
\end{tabular} 

 Mascara 3x3
\end{center} 

\subsubsection{Tipos}
Hay varios tipos de suavizado:
\begin{itemize}
\item Promedio del entorno de vecindad.
\item Filtrado de la mediana.
\item Promedio de la imagen.
\item Suavizado binario de im�genes.
\end{itemize} 

\subsubsection{Promedio del entorno de vecindad}
Es el m�todo que hemos utilizado en el preprocesamiento de las im�genes recibidas por la c�mara.

Se utiliza para la eliminaci�n de ruidos y otros efectos debidos a la cuantizaci�n o a perturbaciones. La raz�n de haber utilizado un suavizado al principio fue para crear un difuminado de la imagen y que futuros filtrados sean mas uniformes. Siempre el objetivo buscado es generar una imagen lo suficientemente simple y objetiva para ser entendida por la red neuronal.
\bigskip 

Las desventajas de este m�todo son que desdibuja contornos y detalles de forma, pero en nuestro caso concreto esto no tiene relevancia.
\begin{center}
\textbf{g(x, y) = $ (\frac{1}{K}) * \Sigma f(n, m) $} 
\end{center} 
(sumatorio de 0 a K, siendo K el numero total de puntos de la vecindad)

M�todo:
\begin{center}
Se utiliza una m�scara de $ 5x5 \rightarrow  wi = 1/25 $
\end{center} 
	 
\begin{figure}[h]
  \centering
  \includegraphics[scale=0.5,bb=0 0 240 160]{arbol1.png}
  \caption{ Imagen original.}
\end{figure}

\begin{figure}[h]
  \centering
  \includegraphics[scale=0.5,bb=0 0 240 160]{arbolSuav.png}
  \caption{ Imagen suavizada por el promedio del entorno de vecindad.}
\end{figure}


%\subsection{Extracci�n de regiones}
\label{regiones_label} 

\subsubsection{Introducci�n}
Las unidades de las im�genes son los p�xeles. Las �nicas propiedades de un p�xel son su posici�n y sus niveles de intensidad.

En las im�genes aparecen ciertas �reas o zonas caracterizadas por el hecho de que constituyen agrupaciones de p�xeles conectados entre s�, pero, adem�s dichos p�xeles presentan caracter�sticas o propiedades comunes, por ejemplo tiene el mismo color. Estas agrupaciones son las regiones.

Nuestro objetivo es binarizar la imagen bas�ndonos en el hecho de que los p�xeles de una determinada regi�n presentan una distribuci�n de intensidad similar, por tanto, a partir del histograma de los niveles en los tres colores, determinamos cual es la zona de dicho histograma y por tanto la regi�n de la imagen.
\bigskip 

\subsubsection{Binarizaci�n por detecci�n de umbral}
Supongamos que el histograma de intensidad

\begin{figure}[h]
  \centering
  \includegraphics[scale=0.5,bb=0 0 280 210]{regoines_histograma.png}
  \caption{ Histograma de intensidades.}
\end{figure}

corresponde a una imagen f(x, y) compuesta por un objeto oscuro sobre un fondo claro, teniendo los p�xeles de un objeto y del entorno intensidades agrupadas en dos tonos dominantes. Una forma obvia de extraer los objetos del entorno es seleccionar un nivel T que separe los niveles de intensidad.

De esta forma un p�xel (x, y) para el cual f(x, y) > T, ser� un p�xel del entorno, en caso contrario ser� del objeto.

Bas�ndonos en esto, podemos considerar la fijaci�n del umbral como una operaci�n que implica pruebas con respecto a una funci�n T como sigue:
\bigskip 

T = T[ x, y, p(x,y), f(x,y)]
\bigskip 

donde f(x,y) es la intensidad en el punto (x,y) y p(x,y) es alguna propiedad local del punto, por ejemplo, la intensidad media de un entorno de vecindad centrado en (x,y). Se creara una imagen binaria g(x,y) definiendo:
\bigskip 

Si $ f(x,y)\geq T $ entonces g(x,y)=255 sino g(x,y)=0 fsi
\bigskip 

Examinando g(x,y) se ve que los p�xeles a los que se asigna el valor 0 corresponden a los objetos, mientras que los que corresponden al entorno tienen valor 255.
\bigskip 

Cuando T depende s�lo de f(x,y), el umbral se llama global. Si T depende tanto de f(x,y) como de p(x,y), entonces el umbral se llama local. Si T depende de las coordenadas espaciales x e y, se llama umbral din�mico.
\bigskip 

\subsubsection{Extracci�n de regiones por el color}
Bas�ndonos en el modelo de color RGB, se pueden extraer de la imagen aquellas regiones en las que predomine una determinada componente de color.

El m�todo consiste en elegir un determinado predicado y determinar en toda la imagen los p�xeles que cumplen dicho predicado. Esos p�xeles los marcamos en blanco y el resto en negro, de esta forma obtenemos una imagen binaria. 

\subsubsection{Selecci�n del umbral �ptimo}
Es dif�cil determinar cual es el umbral optimo para poder llevar acabo una binarizaci�n adecuada. Adem�s debemos tener en cuenta que la iluminaci�n que habr� de unas ocasiones a otras ser� distinta, esto influye en la manera en que la c�mara percibe los colores del entorno, por ejemplo, si hay poca luz los colores ser�n mas oscuros y lo contrario si hubiera mucha luz, por tanto no hemos podido determinar un umbral fijo porque este ser� dependiente del entorno.

La interfaz del pipeline genera para cada modulo de filtro una peque�a ventana que permite la elecci�n de un color de las im�genes que est�n entrando en ese momento por la webCam. De esta manera nos estamos asegurando de seleccionar y fijar el color exacto en esas condiciones del entorno. 

Debido a que las im�genes son a color, al seleccionar un color del entorno, se estar�n fijando autom�ticamente 3 umbrales, uno para el rojo, otro para el verde y otro para el azul.

Los umbrales en nuestros filtros son rangos, se podr�an considerar como un par de umbrales, uno inferior y otro superior, de tal manera que un p�xel (x,y) estar� dentro del rango si:
\bigskip 

Para el color rojo $ \rightarrow T_{inf(rojo)}<f(x,y)_{rojo}<T_{sup(rojo) } $

Para el color verde $ \rightarrow T_{inf(verde)}<f(x,y)_{verde}<T_{sup(verde)} $	

Para el color azul $ \rightarrow T_{inf(azul)}<f(x,y) _{azul} <T_{sup(azul)} $	
\bigskip 

Si las tres componentes del p�xel (x,y) est�n dentro del rango seleccionado, entonces las 3 componentes tomaran el valor blanco (255) y si estan fuera de rango tomaran el valor negro (0). Binarizando as� la imagen.
\bigskip 

Como elegir la tolerancia de este rango. Cuanto mas amplio sea el rango mas cantidad de colores entraran dentro de este. Aunque elijamos el color exacto del entorno que queramos filtrar, si no fijamos bien la tolerancia del rango, la imagen no binarizar� las regiones correctas.
\bigskip 

\begin{figure}[h]
  \centering
  \includegraphics[scale=0.5,bb=0 0 175 175]{regiones_tol_baja.png}
  \caption{ Extracci�n de una regi�n de color de una imagen con tolerancia baja.}
\end{figure}

\begin{figure}[h]
  \centering
  \includegraphics[scale=0.5,bb=0 0 175 175]{regiones_tol_normal.png}
  \caption{ Extracci�n de una regi�n de color de una imagen con tolerancia normal.}
\end{figure}

\begin{figure}[h]
  \centering
  \includegraphics[scale=0.5,bb=0 0 175 175]{regiones_tol_alta.png}
  \caption{ Extracci�n de una regi�n de color de una imagen con tolerancia alta.}
\end{figure}

Para ello la ventana proporcionada por el pipeline para el filtro, no solo permite seleccionar un color determinado de la escena, sino tambi�n elegir en tiempo de ejecuci�n la tolerancia para cada rango (rojo, verde y azul). As� cada vez que el robot cambie de ambiente se puede cambiar desde su interfaz los par�metros de los 6 umbrales. As� quedar�a el color de la regi�n a extraer totalmente acotado y determinado.
\bigskip 

Si ( $ T_{inf(rojo)}<f(x,y)_{rojo}<T_{sup(rojo) }  $ and  $ T_{inf(verde)}<f(x,y)_{verde}<T_{sup(verde)} $  and  
  $ T_{inf(azul)}<f(x,y) _{azul} <T_{sup(azul)} $) entonces

	$ g(x,y)_{rojo} = 255;  $	//Blanco

	$ g(x,y)_{verde}= 255; $	//Blanco

	$ g(x,y)_{azul}= 255; $	//Blanco

sino

	$ g(x,y)_{rojo} = 0;  $	//Negro

	$ g(x,y)_{verde}= 0; $	//Negro

	$ g(x,y)_{azul}= 0; $	//Negro

fsi
\bigskip 

Ejemplo de los 6 umbrales seleccionados en una de las puebas de extracci�n de regiones de color. 
\bigskip 

$ T_{sup(rojo)}  = 170 	T_{inf(rojo) }   = 255 $

$ T_{sup(verde)} =  60	T_{inf(verde)} = 120 $

$ T_{sup(azul)}   =  70	T_{inf(azul) }  = 125 $
\bigskip 

los resultados son estos:
\bigskip 

\begin{figure}[h]
  \centering
  \includegraphics[scale=0.5,bb=0 0 224 168]{regiones_fuente.png}
  \caption{ Imagen original.}
\end{figure}
\begin{figure}[h]
  \centering
  \includegraphics[scale=0.5,bb=0 0 224 168]{regiones_destino.png}
  \caption{ Imagen filtrada.}
\end{figure}


%\subsection{Centrado}
\label{centrado_label} 

Es una de las etapas del procesamiento de im�genes. Recibe como par�metro de entrada no solo la imagen si no un rango de color previamente seleccionado.

Todos los p�xeles cuyo color est� dentro de este rango, formaran un conjunto. El conjunto de las coordenadas cartesianas de estos p�xeles dentro de la imagen.

Vamos a llamar centro de masas (c. m.), a la media de las coordenadas de todos los p�xeles que forman el conjunto, de esta manera el centro de masas ser� la coordenada de un p�xel que puede o no pertenecer al conjunto, pero que representa el centro de la mayor concentraci�n de elementos de este.

La coordenada x ser� la media aritm�tica de todas las coordenadas x de este conjunto y lo mismo con la y.
\bigskip 

$ X = \frac{\Sigma x_{i}}{k} $ , desde i=1..k, siendo k el cardinal del conjunto y $ \forall x_{i} \epsilon conjunto $.

$ Y = \frac{\Sigma y_{i}}{k} $ , desde i=1..k, siendo k el cardinal del conjunto y $ \forall y_{i} \epsilon conjunto $.
\bigskip 

El objetivo es centrar el centro de masas dentro de la imagen. As� estaremos centrando la regi�n de inter�s. El centrado implica un desplazamiento de todos los p�xeles de la imagen.
\bigskip 

Ejemplo:

\begin{figure}[h]
  \centering
  \includegraphics[scale=0.4,bb=0 0 308 201]{centrado.png}
  \caption{ Imagen en blanco y negro. El color elegido ser� el negro.}
\end{figure}

\begin{figure}[h]
  \centering
  \includegraphics[scale=0.4,bb=0 0 308 201]{centrado2.png}
  \caption{ Esta imagen muestra cual ser�a la coordenada que representa el centro de masas del conjunto de p�xeles negros.}
\end{figure}

Seg�n esto, los p�xeles que cumplen la propiedad de estar dentro del rango del color elegido son:

{(6,2),(3,3),(4,3),(5,3),(6,3),(7,3),(8,3),(9,3), (3,4),(4,4),(5,4),(6,4),

(7,4),(8,4),(9,4),(3,5),(3,6),(3,7),(3,8),(3,9),(9,5),(9,6),(9,7),(9,8),(9,9)}

y por tanto el c. m. cae en la posici�n $ (6, 4.84) \cong (6,5) $

siendo el centro de la imagen la posici�n (6,6)
\bigskip 

La posici�n del c. m. no esta en el centro de la imagen. La distancia que hay que desplazar la imagen es el modulo entre el punto central de la imagen y el punto del centro de masas. Por tanto movemos todos los p�xeles de la imagen n posiciones verticalmente y m posiciones horizontalmente para cuadrar el c. m. con el centro de la imagen. En este proceso hay p�xeles de la imagen original que se pierden y otros que se crean y no tienen un valor concreto, estos ser�n creados con el color del entorno, ya que se supone que no son de inter�s.

Bas�ndonos en el ejemplo anterior, el centro de masas esta desplazado respecto al centro de la imagen una posici�n hacia arriba, por lo que hay que desplazar todos los p�xeles de la imagen una posici�n hacia abajo. Esto destruye la fila de abajo y crea una fila nueva en la parte superior a cuyos p�xeles se les asignara el color blanco.
\bigskip 

\begin{figure}[h]
  \centering
  \includegraphics[scale=0.4,bb=0 0 308 201]{centrado3.png}
  \caption{ Imagen que muestra la direcci�n en que se deben desplazar todos los p�xeles.}
\end{figure}

\begin{figure}[h]
  \centering
  \includegraphics[scale=0.4,bb=0 0 308 201]{centrado4.png}
  \caption{ Esta imagen muestra el resultado del desplazamiento.}
\end{figure}

Ejemplo real
%\begin{figure}[h]
 % \centering
 % \includegraphics[scale=0.5,bb=0 0 308 201]{regiones.png}
 % \caption{ Imagen filtrada.}
%\end{figure}

%\begin{figure}[h]
%  \centering
%  \includegraphics[scale=0.5,bb=0 0 308 201]{filtrado.png}
%  \caption{ Imagen filtrada y centrada.}
%\end{figure}

\section{Curiosidad}
El centrado tiene otra utilidad. Debido a que necesita hacer el calculo de cuantos p�xeles cumplen la propiedad de estar dentro del rango de color, si resulta que no hay en toda la imagen ninguno que la cumpla, el m�todo no devuelve la imagen centrada, si no que devuelve NULL. Tal y como esta implementado el pipeline si un modulo saca como estructura de datos un puntero a NULL, las siguientes operaciones que se realizar�an sobre esta estructura dejan de hacerse, esto aumenta la velocidad del programa si no hay gestos o carteles frente al robot, ya que disminuye en gran cantidad el numero de instrucciones realizadas. Se podr�a decir que el centrado es un detector de objetos de inter�s que mantiene al pipeline en \textit{stand by} mientras que no se detecten objetos frente a la c�mara.


%\section{Filtro de Carteles}
\label{filtro_carteles} 

\subsection{Introducci�n}
Los carteles son se�ales al igual que los gestos percibidos visualmente por el robot. Los carteles llevan impresos mensajes, ya sean ordenes, operaciones aritm�ticas o preguntas sobre datos que el robot posea en su base de datos. Este debe ser capaz de desechar toda la informaci�n de la imagen excepto este mensaje.

Es aqu� donde se nos plante� la duda de como resolver este problema, como ense�ar al robot a desechar todos los p�xeles de la imagen excepto aquellos p�xeles negros que forman parte de la regi�n de las letras que forman el mensaje. Hubo distintas soluciones al problema, pero la que mejor ha funcionado es la de crear unos carteles de un color especial, es decir, de un color que no suela encontrarse en el entorno y el mensaje de este impreso en letras negras.

Por tanto ya existe una caracter�stica que diferencia a los p�xeles del mensaje del resto, y es que son lo �nicos p�xeles negros rodeados en todas sus direcciones por p�xeles con la intensidad propia del color especial. En resumen este filtro lo que hace es binarizar la imagen, el objetivo perseguido es parecido al del filtro de los guantes, sen intenta binarizar la imagen dejando a un color lo importante y en otro color lo que no nos interesa. La diferencia es que el procesamiento llevado a cabo en este filtro es mas complejo que en el de los guantes, requiere pasar por mas fases de procesamiento.

As� pues este filtro convertir� im�genes que contengan un cartel con un mensaje X en una imagen blanca con el mensaje negro centrado, horizontal y con el menor numero de ruidos, tambi�n asegura no filtrar la imagen si no se percibe ning�n cartel o si este no esta al 100x100 dentro del campo de visi�n. 

Todo esto son medidas de seguridad para facilitar el futuro an�lisis del mensaje por parte del OCR, evitando posibles fallos de este y que sea lo mas fiable posible.
\bigskip 

Estos son ejemplos del filtro. Las im�genes sin cartel no ser�n procesadas ni pasadas al OCR, para evitar fallos y operaciones innecesarias.
\begin{figure}[h]
  \centering
  \includegraphics[scale=0.5,bb=0 0 253 203]{cartel.png}
  \caption{ Simulaci�n de la captura de un cartel.}
\end{figure}

\begin{figure}[h]
  \centering
  \includegraphics[scale=0.5,bb=0 0 253 203]{cartel2.png}
  \caption{ Resultado de filtrar la imagen.}
\end{figure}

Ejemplos de mensajes dentro de un cartel:
\begin{itemize}
\item Ordenes y par�metros:
	Avanzar MediaAlta, Parar, Girar 90, ...
\item Operaciones aritm�tico-l�gicas:
	(150mod15)/5, (true and false), ...
\item Preguntas a su base de datos:
	Nombres creadores, ...
\end{itemize} 	

\subsection{Fases del filtro}

\subsubsection{Bordes}
En este proyecto hay un factor que siempre tenemos en contra y es el tiempo. Hay que recordar que la captura, procesamiento y an�lisis de las im�genes se hacen en tiempo real, si la captura de im�genes se hace cada X milisegundos hay que asegurarse que todas las operaciones que se deban de realizar tardan menos que ese intervalo. Por tanto los filtros no solo tienen el deber de procesar la imagen si no tambi�n de mejorar el rendimiento del programa.

Por ello son utilizados como sensores de detecci�n de ciertos objetos, en este caso de carteles. 

Esta fase es una funci�n que recibe una imagen de entrada y detecta si la imagen esta completamente dentro del campo de visi�n o solamente en parte, si solo ha sido capturado parte del cartel, el mensaje puede que no haya salido entero por tanto carecer�a de sentido y no seria valido. Si el cartel ha sido capturado en su 100x100 devuelve cierto si no falso. Al devolver falso el resto de operaciones de filtro sobre el cartel se dejan de hacer y se pasa NULL al modulo de OCR para que tampoco realice ninguna operaci�n, mejorando por tanto la eficiencia del programa, ya que solamente realizara operaciones cuando detecte carteles enteros.

La idea para implementar este funci�n es simple, solo hay que buscar p�xeles cuya intensidad este dentro del rango de color especial del cartel si existe alguna regi�n de estos p�xeles en el borde de la imagen capturada implica que el cartel no ha sido capturado en su totalidad.
\bigskip 

\subsubsection{Centrar}
Esto significa centrar el cartel dentro de la imagen. La imagen quedara desplazada lo necesario para que el cartel se encuentre justo en el medio y por tanto tambi�n lo estar� el mensaje. Esto no aporta valor a�adido al posterior an�lisis por parte del OCR, sirve para asegurar la integridad del mensaje en las siguientes fases del filtro, adem�s sirve como al igual que en el filtro de los gesto y como la funci�n anterior de BORDES, como un sensor. El anterior detectaba si el cartel estaba entero y este si existe cartel dentro de la imagen.

El centrado en su implementaci�n cuenta el numero de p�xeles de color especial, es decir, del cartel. Haciendo una media aritm�tica de sus posiciones, por esto si se da el caso en que el numero de p�xeles de este color detectados es igual a cero implica que no hay cartel y por tanto que las siguientes operaciones que se realicen sobre la imagen no tiene sentido al igual que el OCR, por tanto como la funci�n BORDES, devuelve NULL si no hay cartel en la imagen, acelerando la ejecuci�n del programa y si resulta que hay cartel, entonces devuelve la imagen centrada.
\bigskip 

Para saber mas sobre el centrado, \ref {Centrado} (p�gina \pageref{Centrado}).

\subsubsection{Esquinas}
Esta funci�n hace un barrido de la imagen sabiendo ya que existe un cartel en ella, con el objetivo de encontrar la cuatro esquinas de este. Las esquinas son coordenadas cartesianas, cuyo conocimiento es de gran utilidad para realizar una posible rotaci�n del cartel, para la extracci�n de regiones y como sensor.

\begin{itemize}
\item Respecto a la rotaci�n:

Con saber la posici�n de al menos dos esquinas contiguas de las cuatro del cartel podemos conocer cuantos grados esta inclinado el cartel, dentro de la imagen. Ya que dos puntos forman un vector, solo hay que calcular el �ngulo que forma este vector con alg�n eje de coordenadas y sabremos cuanto esta inclinado el cartel, para su posterior rotaci�n.
\item Respecto a la extracci�n de regiones: 

Conocer las esquinas es conocer los limites de lo que nos interesa y de lo que no, ya sabemos que todo lo que este fuera de las esquinas es desechable y lo que queda dentro es necesario procesarlo. 
\item Respecto al sensor:

Conocer al menos tres esquinas, es conocer tres coordenadas y por tanto eso nos permite crear dos vectores. Un vector es un lateral del cartel y el otro la base, sabiendo esto si el �ngulo que existe entre esos vectores es un �ngulo recto, significa que lo que se esta detectando es un cartel, si no cumple esta propiedad es que se ha detectado un objeto del color buscado, pero no es un cartel. Esto evita pasarle al OCR posible informaci�n sin sentido que podr�a generar fallos en el programa principal.
\end{itemize} 

\subsubsection{Rotar}
Como ya hemos dicho el an�lisis siguiente a este procesamiento es llevado a cabo por el OCR, este es un modulo implementado de tal forma que no es sensible al tama�o y en cierta medida al formato de los caracteres, pero si que es sensible cuando estos se encuentran rotados. Muchas veces el robot detectara y procesara carteles que no est�n completamente horizontales, lo que provocar�a fallos en el OCR, por tanto es necesario rotar la imagen lo necesario para que el cartel quede horizontal. Con la funci�n anterior de las esquinas ya sabemos cuantos grados esta girado el cartel, solo hay que pasarle a esta funci�n como par�metros la imagen y los grados a girar.
\bigskip 

Para saber mas sobre la rotaci�n, en \ref {Rotacion} (p�gina \pageref{Rotacion}).

\subsubsection{Extracci�n regiones}
Esta parte es similar a la utilizada en el filtro de gestos. Solo que aqu� no nos interesan las regiones de color especial, si no ciertas regiones rodeadas de este color especial, que en este caso ser�n los caracteres del mensaje.

La extracci�n de regiones binariza la imagen convirtiendo a un color la regi�n buscada y a otro el resto de la imagen, las regiones del mensaje pasaran a negro y el resto a blanco, como en el ejemplo anterior. 

La funci�n ESQUINAS se realizara otra vez despu�s de la rotaci�n (si esta ha sido necesaria) ya que la posici�n de estas dentro de la imagen habr� cambiado. Como ya hemos dicho antes, saber las coordenadas de estas esquinas nos informa de que �rea de la imagen es necesario filtrar y cual directamente es considerada como regi�n no extra�ble y puesta directamente a color blanco, en este caso ser� toda aquella regi�n que quede fuera del cartel.

La extracci�n de regiones propiamente dicha se realizara por tanto dentro del �rea de la imagen perteneciente al interior del cartel. Las regiones que se extraer�n ser�n aquellos p�xeles que cumplan la propiedad de ser oscuros rodeados en todas sus direcciones por color especial. La t�cnica utilizada para la extracci�n de regiones viene explicada en, \ref {Extraccion de regiones} (p�gina \pageref{Extraccion de regiones}).

\subsubsection{Operacion morfol�gica}
Despu�s de todas las operaciones anteriores realizadas sobre la imagen los caracteres del cartel puede que hayan quedado da�ados, es decir, como si se hubieran erosionado, perdiendo suavidad en sus bordes o incluso dejar un mismo car�cter separado en regiones distintas.

Nuestro cerebro tiene la necesidad de encontrar sentido a lo que ve y a unificar figuras inacabadas, pero esta cualidad no la tiene el OCR, si este detecta regiones separadas las tratara como distintos caracteres y buscara el car�cter mas aproximado a estas, lo cual ser�a un error. Es necesario por tanto juntar regiones que hayan quedado separadas y rellenar huecos de los caracteres del mensaje para asegurar por tanto una mejor comprensi�n del mensaje.

La operaci�n morfol�gica realizada sobre la imagen es una operaci�n de cierre, que como ya he dicho rellenara los caracteres.
\bigskip 

Para saber mas sobre estas operaciones, en \ref {Operaciones Morfologicas} (p�gina \pageref{Operaciones Morfologicas}).

\subsection{Pseudoc�digo}
  if( not BORDES and CENTRAR and ESQUINAS)

     ROTAR

     ESQUINAS

     EXTACCION DE REGIONES

     OPERACI�N DE CIERRE

  end if;

\subsection{C�digo}
C�digo disponible en, \ref {Codigo_filtro_carteles} (p�gina \pageref{Codigo_filtro_carteles}).


\chapter{M�dulo de gesti�n de mensajes}

\section{Introducci�n}
Este m�dulo de la aplicaci�n tiene como misi�n recoger todos los mensajes que han podido ser generados por m�dulos de proceso anteriores, y filtrarlos de tal modo que la salida de la l�nea de ejecuci�n est� dotada de cierta coherencia con el resultado deseado. Para ello, admite las entradas en forma de cadenas de texto, y elige cu�les de ellas son las que deber�an dar realmente una salida a los m�dulos posteriores.

\section{Detalles}
\begin{itemize}
  \item {\bf Entrada}: Cualquier cadena de texto, y par�metros para controlar la tolerancia.
  \item {\bf Salida}: Una cadena de texto, la que m�s se asemeja a la que realmente deber�a ser generada.
  \item {\bf Descripci�n}: M�dulo que, a trav�s de la parametrizaci�n, guarda una tabla con las entradas, y, en funci�n de las cantidades de informaci�n, presenta la mejor salida.
\end{itemize}

\section {Arquitectura y funcionamiento del m�dulo}
El m�dulo trabaja con una \emph{tabla hash} inicialmente vac�a. Cuando recibe se�ales procedentes del m�dulo procesador, realiza una de las dos opciones siguientes:
\begin{enumerate}
\item \textbf{El elemento no estaba en la tabla}: Se a�ade a la tabla y se suma una unidad.
\item \textbf{El elemento ya estaba en la tabla}: Se suma una unidad al n�mero de llegadas consecutivas de ese elemento.
\end{enumerate}
Tras este primer paso, se procede a la ``debilitaci�n'' de las otras se�ales. Con esto queremos decir que reducimos el �ndice de refuerzo asociado a todas las se�ales que no fueran la que hemos escogido para que impere, tras una serie de ciclos que parametrizamos a trav�s de los argumentos del m�dulo, la se�al m�s importante (la que m�s veces seguidas ha llegado), que consideramos como la real que deber�a ser transmitida.

De esta forma, siempre mantenemos en la tabla todas las se�ales que van llegando, y creamos un tipo de diagrama de estados borroso, en el que el estado principal es decidido mediante los valores que el m�dulo va asignando a cada registro de la tabla.

Como utilidad a�adida, este m�dulo es uno de los m�s gen�ricos del \emph{pipeline}. El hecho de que los estados posibles se vayan creando de forma din�mica, y que sean s�lo diferenciados por una cadena de texto, ha hecho posible que el m�dulo sea usado para gestionar diferentes l�neas, sin tener que modificar ni una l�nea de c�digo. La �nica parte que hay que personalizar son los par�metros de la instanciaci�n del m�dulo, para que el comportamiento sea lo mejor posible.

\chapter{M�dulo de control del robot}

\section{Introducci�n}
El m�dulo de control de robot ha sido creado con el fin de tener una pieza de software capaz de controlar nuestro robot de una forma sencilla y transparente para los m�dulos que generan la informaci�n de salida. A este m�dulo le llega una estructura de datos que contiene la orden y el par�metro, y el robot se encarga de moverse en funci�n de esa informaci�n. A continuaci�n detallamos c�mo lo hace.

\section{Arquitectura del m�dulo}
El m�dulo en s� tiene una estructura muy simple: consiste en un conjunto de funciones que son exportadas a un \emph{script} programado en el lenguaje \textbf{Lua}, y que funcionan principalmente haciendo llamadas a una librer�a que hemos creado, y que se encarga de controlar el puerto paralelo.

\subsection{Scripts}
% TODO: pues tud�

\subsection{Biblioteca de control del puerto paralelo}
Hemos desarrollado una biblioteca que provee un interfaz lo m�s simple posible de control de los pines del puerto paralelo, y que adem�s es \textbf{multiplataforma}. A trav�s de ella se puede controlar los pines del cable, con llamadas simples, en las cuales s�lo hay que especificar que pin se quiere usar, y si se quiere poner a \emph{alta} o a \emph{baja}. De este modo nos hemos abstra�do, en la implementaci�n del m�dulo en s�, de las peque�as dificultades que puede ocasionar interactuar directamente con la entrada/salida del ordenador.

Para la implementaci�n de esta biblioteca hemos usado a su vez dos bibliotecas externas: \textbf{parapin} para la implementaci�n en GNU/Linux, y la DLL \textbf{inpout32.dll} para el uso en plataformas Win32.

\section{Construcci�n del robot}
Para construir la estructura f�sica del robot, hemos usado piezas de \emph{Lego Technics}. Este material es barato y razonablemente consistente para soportar su propio peso, el de la c�mara, el circuito, y el cable paralelo. Adem�s, destaca principalemente por su versatilidad de uso y su capacidad de reconstrucci�n. El ensamblado de las piezas es inmediato (hay que tener en cuenta que se vende como juguete para ni�os a partir de 12 a�os), y los errores de estructura se pueden subsanar con mucha facilidad.

Sin embargo, hemos escogido este material por la facilidad que hemos encontrado para generar m�quinas m�viles. Los conjuntos de piezas de \emph{Lego Technics} suelen venir acompa�ados por estructuras m�s complejas de construcci�n, como motores y brazos hidr�ulicos, elementos que hemos usado para nuestro robot.

El circuito ha sido conectado sobre una placa peque�a de entrenador, a pesar de su fiabilidad relativa, es r�pida de montar y de depurar. Hemos usado un chip \textbf{L293B}, que sirve para control de motores bidireccionales de corriente cont�nua, y alimentamos el circuito con una pila de 9 voltios. El circuito que usa el robot es el siguiente:

% TODO: poner la imagen del circuito.

\section {Fotos}
A continuaci�n mostramos algunas fotos del robot que hemos construido:
% TODO: pues tud�

%\begin{figure}
%  \centering
%  \includegraphics[bb=0 0 88 66]{robot.png}
%  \caption{Foto del robot}
%\end{figure}


\section {C�digo}
% TODO: poner bien la referencia
Ver anexo: documentaci�n del m�dulo de robot.c, en \ref {robot_8c} (p�gina \pageref{robot_8c}).
\chapter{Procesamiento de texto reconocido}

\section{Introducci�n}
Una vez que el OCR ha reconocido una cadena de texto y la ha validado, env�a la se�al a este m�dulo, que se encarga de procesarlo, y crear una nueva salida en el formato propio del pipeline para el entorno 3D y el robot.

\section{Detalles}
\begin{itemize}
  \item {\bf Entrada}: Una cadena ya preparada del OCR.
  \item {\bf Salida}: Un resultado en forma de cadena de respuesta o de orden para el robot.
  \item {\bf Descripci�n}: M�dulo que, a partir de una cadena ya reconocida, calcula un resultado matem�tico, o da �rdenes de conducta a un robot.
\end{itemize}

\section{Implementaci�n}

Para implementarlo, hemos usado un programa en \textbf{Prolog}, que est� formado en su mayor parte por una DCG \footnote{Definite clause grammar}. Este programa recibe la entrada del m�dulo, y, tras procesarla, crea una salida, que puede ser una cadena de respuesta, el resultado de una operaci�n aritm�tica (soporta par�ntesis y anidamiento de expresiones) o una orden para la salida.

\section{Ampliabilidad}

La idea de crear el m�dulo como antes hemos explicado ha sido propiciada por la idea de que la ``inteligencia del robot'' fuese algo ampliable. Con s�lo modificar este archivo, se puede dotar a la m�quina de m�s informaci�n (ampliando el diccionario), o de m�s capacidad de proceso (creando m�s reglas de reconocimiento en la DCG).

\chapter {Otros m�dulos}

\section{Introducci�n}
A continuaci�n explicamos el funcionamiento del resto de los m�dulos. Los aglutinamos en esta secci�n, por carecer de inter�s la explicaci�n detallada de los mismos, debido a su sencillez.

% TODO: poner las referencias

\section {M�dulo de post-gesti�n}
Hemos a�adido un m�dulo que recoja todas las se�ales que son generadas por las distintas ramas de proceso del \emph{pipeline}, y las agrupe en una sola salida v�lida, para ofrecer de este modo, a trav�s de sus puertos, datos a todos los m�dulos de salida.

Adem�s, hemos incluido en el m�dulo la posibilidad de generar salidas en un script, para que la depuraci�n de los m�dulos no dependa de los que est�n m�s arriba en el grafo, y tener m�s potencia de control de errores.

Puede examinar su documentaci�n de c�digo en \ref{caca}.

\section{Ventana de par�metros}
La ventana de par�metros es una interfaz gr�fica en la que es posible elegir un color, y tres tolerancias (una para el rojo, otra para el verde y otra para el azul), y alimentar as� a un m�dulo de filtro, de tal modo que la parametrizaci�n de los valores del mismo se pueda hacer en tiempo real.

Puede examinar su documentaci�n de c�digo en \ref{caca}.

\section{Ventana de im�genes}
La ventana de im�genes es simplemente un m�dulo que muestra gr�ficamente, en una ventana que se redimensiona autom�ticamente, una imagen que sigue el formato propio de nuestra aplicaci�n (ver secci�n \ref{formato_imagenes}).

La ventana de im�genes tiene como funci�n a�adida la capacidad de guardar una toma de la imagen que se este dibujando en el instante en el que se pulsa \textbf{F5}.

Puede examinar su documentaci�n de c�digo en \ref{caca}.

\section{M�dulo de salida}
Este m�dulo es simplemente una ventana con un cuadro de texto que es capaz de imprimir en �l todas las cadenas (en formato C, \verb|char *|) que le llegan a trav�s de sus puertos.

Puede examinar su documentaci�n de c�digo en \ref{caca}.


\part{Interfaz del programador}
\label{anexo2}

\end{document}
