\chapter{Dise�o}

\section{Introducci�n}

\subsection {Arquitectura de la aplicaci�n}
Al dise�ar la aplicaci�n escogimos implementar un sistema altamente modular para conseguir un alto grado de independencia en el desarrollo y de facilidad de ampliaci�n. Por esta raz�n hemos invertido parte del trabajo en desarrollar una plataforma propia de enlace de m�dulos, en la que la una parte de la aplicaci�n (lo que hemos denominado {\em pipeline} o {\em tuber�a}), se encargue de realizar el trabajo mec�nico, que, b�sicamente, se compone de:
\begin{itemize}
\item Conectar los m�dulos mediante puertos independientes con diferente informaci�n por puerto, pudiendo crear conexiones {\em 1 a 1}, {\em n a 1}, {\em 1 a n}, y {\em n a n}.
\item Iniciar y cerrar los m�dulos, creando y liberando la memoria necesaria y llamando a las funciones pertinentes de cada m�dulo.
\item Gestionar un reloj de ciclos de ejecuci�n, transmitiendo la acci�n por el grafo que forma la arquitectura de m�dulos.
\item Control de proyectos de aplicaci�n din�micos, mediante definici�n de los mismos en {\bf XML}. De esta forma, diferentes archivos de configuraci�n de proyecto puden crear aplicaciones totalmente distintas sin tener que reprogramar nada.
\item Manejo de errores mediante retrollamadas a funciones definidas por el usuario.
\end{itemize}
El {\em pipeline} es multiplataforma y funciona con m�dulos compilados desde {\em cualquier lenguaje est�ndar} como bibliotecas din�micas. Esto dota a la aplicaci�n de un marco muy amplio de uso en cualquier �mbito de desarrollo.
