\chapter{M�dulo de gesti�n de mensajes}

\section{Introducci�n}
Este m�dulo de la aplicaci�n tiene como misi�n recoger todos los mensajes que han podido ser generados por m�dulos de proceso anteriores, y filtrarlos de tal modo que la salida de la l�nea de ejecuci�n est� dotada de cierta coherencia con el resultado deseado. Para ello, admite las entradas en forma de cadenas de texto, y elige cu�les de ellas son las que deber�an dar realmente una salida a los m�dulos posteriores.

\section {Arquitectura y funcionamiento del m�dulo}
El m�dulo trabaja con una \emph{tabla hash} inicialmente vac�a. Cuando recibe se�ales procedentes del m�dulo procesador, realiza una de las dos opciones siguientes:
\begin{enumerate}
\item \textbf{El elemento no estaba en la tabla}: Se a�ade a la tabla y se suma una unidad.
\item \textbf{El elemento ya estaba en la tabla}: Se suma una unidad al n�mero de llegadas consecutivas de ese elemento.
\end{enumerate}
Tras este primer paso, se procede a la ``debilitaci�n'' de las otras se�ales. Con esto queremos decir que reducimos el �ndice de refuerzo asociado a todas las se�ales que no fueran la que hemos escogido para que impere, tras una serie de ciclos que parametrizamos a trav�s de los argumentos del m�dulo, la se�al m�s importante (la que m�s veces seguidas ha llegado), que consideramos como la real que deber�a ser transmitida.

De esta forma, siempre mantenemos en la tabla todas las se�ales que van llegando, y creamos un tipo de diagrama de estados borroso, en el que el estado principal es decidido mediante los valores que el m�dulo va asignando a cada registro de la tabla.

Como utilidad a�adida, este m�dulo es uno de los m�s gen�ricos del \emph{pipeline}. El hecho de que los estados posibles se vayan creando de forma din�mica, y que sean s�lo diferenciados por una cadena de texto, ha hecho posible que el m�dulo sea usado para gestionar diferentes l�neas, sin tener que modificar ni una l�nea de c�digo. La �nica parte que hay que personalizar son los par�metros de la instanciaci�n del m�dulo, para que el comportamiento sea lo mejor posible.

\section {C�digo}
% TODO: ver la etiqueta buena de esto
Ver anexo: documentaci�n del m�dulo de gestion.c, en \ref {gestion_8c} (p�gina \pageref{gestion_8c}).
