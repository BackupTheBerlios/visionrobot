\chapter{M�dulo de control del robot}

\section{Introducci�n}
El m�dulo de control de robot ha sido creado con el fin de tener una pieza de software capaz de controlar nuestro robot de una forma sencilla y transparente para los m�dulos que generan la informaci�n de salida. A este m�dulo le llega una estructura de datos que contiene la orden y el par�metro, y el robot se encarga de moverse en funci�n de esa informaci�n. A continuaci�n detallamos c�mo lo hace.

\section{Arquitectura del m�dulo}
El m�dulo en s� tiene una estructura muy simple: consiste en un conjunto de funciones que son exportadas a un \emph{script} programado en el lenguaje \textbf{Lua}, y que funcionan principalmente haciendo llamadas a una librer�a que hemos creado, y que se encarga de controlar el puerto paralelo.

\subsection{Scripts}
% TODO: pues tud�

\subsection{Biblioteca de control del puerto paralelo}
Hemos desarrollado una biblioteca que provee un interfaz lo m�s simple posible de control de los pines del puerto paralelo, y que adem�s es \textbf{multiplataforma}. A trav�s de ella se puede controlar los pines del cable, con llamadas simples, en las cuales s�lo hay que especificar que pin se quiere usar, y si se quiere poner a \emph{alta} o a \emph{baja}. De este modo nos hemos abstra�do, en la implementaci�n del m�dulo en s�, de las peque�as dificultades que puede ocasionar interactuar directamente con la entrada/salida del ordenador.

Para la implementaci�n de esta biblioteca hemos usado a su vez dos bibliotecas externas: \textbf{parapin} para la implementaci�n en GNU/Linux, y la DLL \textbf{inpout32.dll} para el uso en plataformas Win32.

\section{Construcci�n del robot}
Para construir la estructura f�sica del robot, hemos usado piezas de \emph{Lego Technics}. Este material es barato y razonablemente consistente para soportar su propio peso, el de la c�mara, el circuito, y el cable paralelo. Adem�s, destaca principalemente por su versatilidad de uso y su capacidad de reconstrucci�n. El ensamblado de las piezas es inmediato (hay que tener en cuenta que se vende como juguete para ni�os a partir de 12 a�os), y los errores de estructura se pueden subsanar con mucha facilidad.

Sin embargo, hemos escogido este material por la facilidad que hemos encontrado para generar m�quinas m�viles. Los conjuntos de piezas de \emph{Lego Technics} suelen venir acompa�ados por estructuras m�s complejas de construcci�n, como motores y brazos hidr�ulicos, elementos que hemos usado para nuestro robot.

El circuito ha sido conectado sobre una placa peque�a de entrenador, a pesar de su fiabilidad relativa, es r�pida de montar y de depurar. Hemos usado un chip \textbf{L293B}, que sirve para control de motores bidireccionales de corriente cont�nua, y alimentamos el circuito con una pila de 9 voltios. El circuito que usa el robot es el siguiente:

% TODO: poner la imagen del circuito.

\section {Fotos}
A continuaci�n mostramos algunas fotos del robot que hemos construido:
% TODO: pues tud�

%\begin{figure}
%  \centering
%  \includegraphics[bb=0 0 88 66]{robot.png}
%  \caption{Foto del robot}
%\end{figure}


\section {C�digo}
% TODO: poner bien la referencia
Ver anexo: documentaci�n del m�dulo de robot.c, en \ref {robot_8c} (p�gina \pageref{robot_8c}).